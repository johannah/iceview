%% LyX 2.0.5.1 created this file.  For more info, see http://www.lyx.org/.
%% Do not edit unless you really know what you are doing.
\documentclass[12pt,english]{report}
\usepackage[T1]{fontenc}
\usepackage[latin9]{inputenc}
\usepackage{babel}
\usepackage{longtable}
\usepackage{float}
\usepackage{calc}
\usepackage{amsthm}
\usepackage{amsmath}
\usepackage{setspace}
\usepackage[unicode=true,pdfusetitle,
 bookmarks=true,bookmarksnumbered=false,bookmarksopen=false,
 breaklinks=false,pdfborder={0 0 1},backref=false,colorlinks=false]
 {hyperref}

%graphicx is not part of included utsa thesis
\usepackage{graphicx}
%\usepackage{subcaption}
\graphicspath{{./figures/}}
\makeatletter

%%%%%%%%%%%%%%%%%%%%%%%%%%%%%% LyX specific LaTeX commands.
\providecommand{\LyX}{\texorpdfstring%
  {L\kern-.1667em\lower.25em\hbox{Y}\kern-.125emX\@}
  {LyX}}
%% Because html converters don't know tabularnewline
\providecommand{\tabularnewline}{\\}
\floatstyle{ruled}
\newfloat{algorithm}{tbp}{loa}[chapter]
\providecommand{\algorithmname}{Algorithm}
\floatname{algorithm}{\protect\algorithmname}

%%%%%%%%%%%%%%%%%%%%%%%%%%%%%% Textclass specific LaTeX commands.
\usepackage[ms]{UTSAthesis}      
\usepackage{times}            
\usepackage{latexsym}

\newenvironment{ruledcenter}{%
  \begin{center}
  \rule{\textwidth}{1mm} } {%
  \rule{\textwidth}{1mm} 
  \end{center}}%


  \theoremstyle{definition}
  \newtheorem{defn}{\protect\definitionname}
\theoremstyle{plain}
\newtheorem{thm}{\protect\theoremname}

\@ifundefined{showcaptionsetup}{}{%
 \PassOptionsToPackage{caption=false}{subfig}}
\usepackage{subfig}
\makeatother

  \providecommand{\definitionname}{Definition}
\providecommand{\theoremname}{Theorem}

\begin{document}
\bibliographystyle{acm}
%\committee{Sos Agaian, Ph.D., Chair}{Prof Hanumant Singh, Ph.D.}{Prof. C, Ph.D.}{Prof. D, Ph.D.}{Prof. E, Ph.D. }
\supervisor{Sos Agaian, Ph.D.}


\cosupervisor{Hanumant Singh, Ph.D.}



\committeeC{Prof A}


\committeeD{Prof A}


\committeeE{Prof A}

\informationitems{Master of Science in Electrical Engineering}{M.Sc.}{Department of Electrical Engineering}{College of Science}{December}{ 2016}


\thesiscopyright{Copyright 2016 Johanna Hansen \\
All rights reserved. }


\dedication{\emph{I would like to dedicate this thesis to the many mentors and 
teachers who have guided me throughout this process.}}


\title{\textbf{AN AUTOMATED SYSTEM FOR HIGH RESOLUTION SEA ICE IMAGE  SEGMENTATION AND MOSAICING }}


\author{Johanna Hansen}

\maketitle

\begin{acknowledgements}

I would like to thank my family, friends, colleagues, and mentors for their 
unwavoring support of my efforts. 

I would also like to thank the UTSA Graduate School for reviewing my work.   
I especially thank Ms.  Monique Wimberly for corresponding with me over email and for providing guidance while I was working remotely. 

\end{acknowledgements}

\begin{abstract}
The polar regions play an outsized role in the future of travel, commerce, and 
defense as sea ice retreats in a changing climate. This steady retreat is opening 
up ocean passageways and resources for the first time in modern history. 
Despite these changes, traveling in ice-laden waters remains dangerous.  
Current sensing and forecasting capabilities are unable to predict precise movement and characterization of sea ice. Unmanned Aerial Vehicles (UAVs) present the most promising technology to improve a vessel's situational awareness when compared to traditional sensors. 
The primary goal of this thesis is to enable the immediate analysis and utility of high resolution aerial imagery captured by UAVs. We acomplish this by  presenting a robust approach to mosaicing large-scale ice imagery using a probabilistic filtering approach and an automated method for ice/water segmentation. 
Although image mosaicing has become a fairly routine process with modern computers and common scenes, traditional 
image mosaicing algorithms fail when tasked  with scenes of sea ice. 
The featureless nature of solid pack ice and the dynamic nature of 
the sea make it nearly impossible to determine image overlap without our contributions. 
In addition to enabling us to calculate sea ice concentration, our ice segmentation technique allows us to mask sea pixels for mosaicing. Along with probabilistic filtering that integrates  
sensor information from the UAV to narrow the search space for image overlap, this
masking significantly reduces false matches in our mosaicing routine. 
In addition to these contributions, we produce a provide a new baseline for 
sea ice segmentation and compare feature descriptors for our dataset.  
Our results are demonstrated using 
data obtained over several days on an Arctic expedition. 

\end{abstract}

\pageone{}
\chapter{Introduction}
%This work is useful for scientists
%seeking to better understand the world's changing climate and 
%for commercial  ships navigating in ice laden waters.  
%Though UAVs provide an excellent data collection platform, 
%that data can be labor intensive to process and interpret because the  images captured provide only a small field-of-view. 
%%This means that for a typical scouting mission, hundreds, if not thousands of 
%%images are collected. 
%The common way to organize 
%the hundreds or thousands of images collected in a flight is by joining the overlapping images into a single field of
% known as a mosaic. The process of creating a mosaic automatically requires 
%finding overlapping scenes within neighboring images and then registering each image onto the mosaic. 
%


\section{Motivation}
%Sea ice covers about 7\% of the world's oceans \cite{NOAA}, maintainin a heat sink that has a large impact on the global climate \cite{Sephton94}.  Sea ice forms when seawater freezes and rises to the ocean surface or when land-residing glaciers calve into the sea. 
Nearly 7\% of the worlds oceans \cite{NOAA} are covered with sea ice each winter. 
The extent of the ice reliably grows both in extent and thickness each winter, and 
then recedes in the summer.  
In recent years, however, the median extent of the summer Arctic sea ice has decreased 
enough that more ships are venturing into the waters. In the summer of 2012, ice
receded to its lowest recorded level of 3.41 million square km, down from 
the 1979-2000 average of about 7 million square km \cite{NewNormal}. 
 A record 30 vessels travelled through the Northwest Passage in the summer of 2012 \cite{state}. 
The previously inaccessable cold waters of the Arctic harbour rich natural resources 
that many are 
eager to tap into, positioning the region  
to play an outsized role in the future of resource extraction, research, commerce, and 
defense. 
Despite the retreating ice, expeditions into the world's frozen oceans require expert navigation, 
specialized equipment, and 
real-time knowledge of the constantly shifting conditions. 
Ice floes and icebergs  
can incapacitate ships, strand mariners, and cause environmental disasters.
Real-time knowledge about the ice 
conditions is necessary for a ship working in polar regions. 
This surveillance 
still presents  a technological challenge and requires significant expenditures. 

Historically, ships working in icy conditions
have relied on human scouts, marine radar, helicopters, and/or satellite imagery to monitor 
the sea ice movement.  However, as detailed in Table \ref{tab:compare}, 
these approaches are resource intensive in the 
case of helicopters, limited in view in the case of radar, and suboptimal in the case of satellites. 
Manned aircraft are expensive to operate, requiring 
expert pilots and precious deck space. They also do not necessarily solve the 
problem of interpreting the sea state in a manner which can be fed into ship control systems. 
Severe weather or tumultuous seas can limit when a manned aircraft can be safely 
deployed. Often these times are when it is most important retrieve local conditions. 
Satellite sensors can provide useful
overviews, but the data captured is low resolution both in time and coverage. It 
is also prone to occlusion by cloud cover or atmospheric conditions in the case of 
of passive sensors and speckle noise in the case of active sensors. 


\begin{table}[h]
\caption{Brief description  and comparison of sea ice sensing techniques.  }
\centering \noindent
            %\begin{tabular}{| l | l | l | l |}
\begin{tabular}{| p{1.7cm} | p{4.7cm} | p{4.7cm} | p{4.7cm} |}
    \hline
    \textbf{Technique} & \textbf{Description} & \textbf{Shortcomings } & \textbf{Advantages} \\
     \hline
     Helicopter & Helicopters can be carried on ships for reconassance with imaging sensors and human scouts. & Expensive, need trained pilot on board, large deck footprint, does not solve automated data analysis and DP integration problem, dangerous in bad weather. & Relatively high temporal resolution (cost and safety dependent), human decision making, doubles as evacuation vehicle.  \\
\hline
Satellite & Passive imagery or active radar sensing performed by a handful of satellites & Low spatial and time resolution  (500-1000 m per pixel resolution captured every 1 to 2 days in the case of MODIS \cite{MODIS} or 100 m per pixel resolution for Radarsat-2 repeated daily in the Arctic \cite{CanadianIceService}) make it imprecise for tracking local floes. Can be effected by weather and cloud cover and SAR is prone to speckle noise.  & Wide coverage allows relatively long-term planning, radar can penetrate fog and clouds and is not dependent on daylight.  \\
\hline
Marine Radar & Rotating active scanner mounted on the masts of ships with coverage of approximately 10m to 1km  & Highest resolution in time, low imaging resolution makes icebergs  which float low in the water difficult to distinguish from less dangerous ice. & Constant sensing for immediate information, some systems already integrate with Dynamic Positioning, is a technology that mariners are familiar with, can penetrate through fog \\
\hline
UAV & Unmanned platform for high resolution, medium range sensing & Difficult to digest and integrate data, new technology for mariners & Low-cost and relatively expendable tool, safe for crew, complete control of sensing temporal frequency, ineffective in fog or rain \\
\hline


\end{tabular}
\label{tab:compare}
\end{table}




\begin{figure}
     \centering
\includegraphics[width=.9\linewidth, height=0.6\linewidth]{UAVICEdiagram.png}
  \captionsetup{width=.9\linewidth}
  \captionof{figure}{
  Block diagram of proposed sea ice mosaic system. }
    \label{fig:diagram}
\end{figure}

Low-cost UAVs can fill the gap between wide-area satellite coverage and
reconnaissance by helicopter. Their platforms are nearly disposable when compared to the 
competing sensors. In 
bad weather, the data may be retrieved remotely, even if the vehicle 
is unable to return to the ship. 
The  main
advantage over satellite imagery is that the 
data is captured at a much higher spatial resolution and can be gathered at the
operator's discretion. 
UAVs require little expertise to operate and little deck space. Because 
there is no human observer of the terrain, the data collected from the vehicle must 
be processed and analyzed in a way that is useful for non-technical decision makers. 
Capturing images from a low-altitude provides detailed information of the 
target scene, but the trade off for resolution is field-of-view. 
Although UAVs can capture images with high resolution, each image only covers a 
small portion of the survey, making it difficult for humans to interpret the
hundreds if not thousands of images that result from a single survey mission. 
Despite being an ideal platform for sea ice reconnaissance, the image data collected 
by UAVs must be processed in order to make it  useful for mariners or scientists 
seeking attempting to gain greater situational awareness.

\section{Problem Formulation}
% This means that a single flight can result in thousands 
%of images that must be organized into a human-readable format. 
%We propose to convert these images into a single mosaic of the scene. The process 
%of building a mosaic involves registering images that contain overlapping 
%views of the same scene and pasting them onto a common frame. 
%
%In this work, we 
%tackle mosaicing for datasets which remains challenging for conventional pipelines 
%and add new capability which automatically classifies sea ice 
%type at the pixel level. 


\begin{figure}
    \centering
    \begin{minipage}{.5\textwidth}
        \centering
        \includegraphics[width=.9\linewidth]{no_features.jpg}
        \captionsetup{width=0.9\linewidth}
        \captionof{figure}{A typical image of sea ice captured from a UAV with few features and homogeneous texture. }
         \label{fig:no_features}
     \end{minipage}%
     \begin{minipage}{.5\textwidth}
         \centering
        \includegraphics[width=.9\linewidth]{good_features.jpg}
        \captionsetup{width=0.9\linewidth}
        \captionof{figure}{A sea ice image with many corners and distinctive  features to use for feature matching. }
             \label{fig:good_features}
      \end{minipage}
\end{figure}



To obtain understanding 
of a scene spread across many high resolution images with a small field-of-view, 
it useful to combine the information into a larger image by combining the many images 
together into a single overview image called a mosaic. 
Currently, there is no practical or robust method of generating an image mosaic 
of high resolution, large-scale sea ice imagery. 
Although we have information about where each image was captured based on data 
taken from the UAV this is rarely accurate enough to provide precise pixel-wise alignment. 
Instead, most high resolution techniques rely on the images themselves to 
provide information on alignment.
Several commercially available programs 
including AutoPano \cite{AutoPano}, Hugin \cite{Hugin}, and PhotoScan \cite{Photoscan}
have made the mosaicing process almost routine for common scenes. However, these 
software packages
have been applied to our sea ice dataset with disappointing results. 
We believe that stitching and warping errors in these mosaics resulted from incorrect feature 
matching or the projection of object height as a result of utilizing projective transformations. 
Sea ice images present a 
fairly unique challenge in that the images contain largely repeating, homogeneous 
pixels in ice which contain little information for matching images. In addition, 
features extracted from water are necessarily dyanmic across our images which are 
captured at distinct points in time. These water features will never match across 
corresponding scenes and invalidate routine mosaicing procedures. 

In order to 
produce an acceptable overview image from high resolution ice imagery, one must overcome 
poor and dynamic features. We propose to do this by first segmenting ice from water 
in the images, thus masking feature extraction in known dynamic regions. Then, 
to overcome areas the lack of distinctiveness of sea ice, we utilize probabilistic 
filtering techniques that captilize on sensor data from the UAV to minimize image 
overlap search space. The combination of both of these techniques allows us to 
produce a valid scene overview of high-resolution sea ice imagery. 
In addition we provide sea ice concentration calculations as a result of ice/water segmentation. 
As an important step in developing our pipeline, we also assess the effectiveness 
of several popular feature descriptors on sea ice imagery. 

\section{Thesis Overview}

This thesis contains seven chapters in which we describe new techniques for  
both segmentation and image mosaicing of high resolution sea ice imagery. 
We break the problem down into 3 distinct phases in which we provide background, 
describe problems with existing approaches, and present our innovations: 

\begin{itemize}
    \item{Chapter \ref{chap:ice}: Pixelwise Segmentation of Ice/Water}
    \item{Chapter \ref{chap:feat}: Feature Descriptor Performance Evaluation}
    \item{Chapter \ref{chap:mosaic}: Topology estimation using Bayesian Estimation for Image Mosaicing}
\end{itemize}

In Chapter \ref{chap:results}, we provide information about our new dataset and present 
mosaic results. Finally, in the conclusion, Chapter \ref{chap:conclusion}, a summary and goals 
for future work is presented. 

We assume that the readers have basic knowledge of digital photography and 
aerial vehicles, however basic concepts are summarized in Chapter \ref{chap:background} before 
delving into more complicated computational concepts. 
We approach sea ice not from the perspective of an ice expert or scientist, 
but as an engineer attempting to maximize the utility of our tool for experts. 

\section{Contribution of Thesis}

In this thesis, a new automated system
for mosaicing high resolution sea ice
imagery is presented. 
This process, described by Figure \ref{fig:diagram} is an effort to enable 
the effective use of UAVs for sea ice reconnaissance. 
We present a new method for segmenting sea ice images into 
ice and water masks, thereby removing dynamic parts of the scene contained in the 
water enabling image matching. This step also  provides a basis for calculating ice concentration. 
We also present a method to 
use a probabilistic filtering approach  to track features through distinct images 
using metadata collected by the flight vehicle.
This reduces false image matches when compared to image-only mosaicing approaches, 
and improves the global consistency of the output mosaic,
especially when faced with homogeneous ice-only images with few features. 
In addition, we evaluate this unusual dataset against existing feature descriptors to 
provide a baseline for sea ice performance. 

\section{Related Work}
\label{sec:related}
UAVs are rapidly becoming the medium of choice for performing sea ice
reconnaissance as indicated by several recent publications \cite{Lagunov14, Billington15} 
describing their potential utility.  As far as we know, 
we are the first to implement large-scale sea ice mapping on low altitude data. 
In the following paragraph, we discuss 
similar work in ice density estimation, iceberg tracking, and ice segmentation 
with low altitude sea ice imagery. 

The most similar work was published in a 2015 thesis  by Flaten \cite{Flaten-thesis15}. 
Flaten described potential ice-monitoring  roles for UAVs in Arctic operations 
and described appropriate platform performance capabilities for performing 
occupancy-grid based ice density estimation based on thermal images with 
Otsu thresholding used for sea/ice segmentation. Unfortunately, 
the project was unable to work real data, and instead relied on simulated images 
to test ideas. The author relied on navigation sensors to provide overlap information for 
density calculations and though the results on simulated data was visually 
appealing, they did not produce satisfactory accuracy.
In addition, Flaten did not produce an overview mosaic, a product that we 
think is crucial for the success of UAV reconnaissance missions. 

In \cite{Skjetne14}, the author from the Norwegian University of Science and Technology 
present large project to develop a UAV based approach 
for monitoring sea ice and automatically integrating the information into 
drift prediction to aid ships 
attempting to maintain Dynamic Positioning (DP) in ice-laden waters.  
Dynamic Positioning is the name for the control system which utilizes sensors 
such as GPS receivers, IMUs, and marine radars to calculate the ship's 
current position and
initiate the necessary actuation of the  
propulsion and steering systems to move the ship to a prescribed location.
DP is  a vital part of modern ship navigation and is especially 
important for drilling or scientific operations in which the ship must maintain a 
static position for long periods of time. Information about real-time ice movement 
 is essential to the proper autonomous function of DP systems in icy waters,
but lacking in most systems. We think that integration with DP systems is important 
for a successful ice reconnaissance framework, though it is out of the scope of this 
work. As part of the project, Haugen and Imsland published  \cite{Haugen14} in 2014. 
This work divide ice monitoring with UAVs into two problems; 
dynamic sea ice monitoring to track the movement 
of large bodies of ice sheets and  tracking of large icebergs, 
over time. In this work, the present a path planning framework for UAV missions 
to collect data needed to integrate into the tracking system. 
Zhang has tackled the project by processing image and sensory data  
for ice observation. In \cite{Zhang13}, Zhang et al. automatically segment 
sea/ice floes using Otsu thresholding and the watershed algorithm. They expand 
upon this work in \cite{Zhang14} and \cite{Zhang15floe}
to use k-means clustering and Gradient Vector Flow Snakes to obtain 
ice concentration and classify ice types, but fail to provide quantifiable results. 

% note Patil16 looks plagerized 

\chapter{Background}
\label{chap:background}
Both segmentation and image mapping have been studied extensively, but there is  not yet a 
method which works on high resolution sea ice imagery. In this chapter, we develop 
an essential background in sea ice and in remote sensing. We explore 
related  sea ice segmentation techniques in Section \ref{sec:seg}. 
Finally, in Section \ref{sec:map}, we discuss image mapping techniques which tackle 
similar problems as those in our work. 


\section{Sea Ice}
\label{sec:ice}
Ice thickness and its spatial extent vary greatly depending on seasonal 
weather and climate. 
Each autumn, freezing temperatures cause a thin sheet of surface 
ice to blanket the poles. This newly formed ice, called first-year ice, 
 thickens throughout the winter and then usually melts back into 
the sea by summer. If the first-year ice manages to survive the first summer, 
it becomes thicker and more dense as the years 
progress and becomes known as \emph{multiyear ice}.
In addition to sea ice that forms in the ocean, icebergs are added to the sea when they 
calve off of land-based glaciers.  
In the Arctic, much of the sea ice persists year 
after year, forming dense and treacherous multi-year ice capable of piercing holes in 
even the strongest icebreaking vessels \cite{ArcticCoverage}.

Many governments and organizations dedicate significant resources to providing 
timely assessments and maps of sea ice coverage. 
One of the most public datasets for sea ice is produced by the Canadian Ice Service. 
Their team of experts produce daily ice charts using data collected from 
airplanes and satellites. 
The National Snow and Ice Data Center, funded by the United States government, 
also manages and distributes scientific data, research, and tools for analysis of sea ice \cite{NSIDC}. 

The Manual of Ice (MANICE) provided by the Canadian Ice Service, describes various 
types of ice discussed throughout this paper \cite{CanadianIceService}. Below we provide 
a brief introduction to several forms of ice relevant to our experiments.  
\begin{itemize}
    \item{\textbf{Pancake ice} is defined as predominately 
circular pieces of ice which are 30 cm to 3 m in diameter. The edges are typically 
raised from the pieces striking against one another. See Figure \ref{fig:pancake} for reference. }
\item{An \textbf{ice cake} is a flat piece that is less than 20 m across. See Figure \ref{fig:cake} for reference. }
\item{An \textbf{ice floe} is a flat piece of ice that is 20 m or more across. Small floes are 
    less than 100 m and giant floes are those pieces that are greater than 10 km across. }
\item{\textbf{Slush} is recently formed ice which is saturated with water and composed of crystals that are only weakly frozen together.} 
\item{\textbf{Glacial ice} is ice that originated from a land based glacier. When glacial ice is floating on the sea, it is broken up into \textbf{icebergs},  \textbf{Bergy bits}, and \textbf{growlers} which describes level at which the ice floats in the water.  Icebergs are massive pieces of ice that protrude more than 5 m above the sea. Bergy bits are pieces of icebergs which show 1 to 5 m while growler  show less than 1 m above the water line}. 
%\item{\textbf{Multi-year ice} has survived at least two summer's melt that is nearly salt-free and blue in color.}
\end{itemize}

Periods of poor visibility due to fog and snow make detecting ice visually difficult, 
even in the summer months which experience near constant daylight. 
For ships travelling in polar regions, it is important to differentiate 
between first-year ice and multi-year ice which can be dangerous.  
From an ice management perspective, floe size and type distribution are important parameters
for estimating the melting rate and understanding how to anticipate ice-breaking needs.  
Despite its importantance, there exist few resources for  automatically analyzing 
local observations of sea ice.  

When assessing ice, Sephton et al. in \cite{Sephton94} outlines the following qualities that are of significance:
\begin{itemize}
    \item{The type, thickness,  position, and size of discrete ice floes}
    \item{Total concentration of sea ice as defined by the fraction of the surface area covered by ice over the measured region. }
    \item{Local and global movement of ice floes}
\end{itemize}

\begin{figure}
    \centering
    \begin{minipage}{.5\textwidth}
        \centering
        \includegraphics[width=.9\linewidth]{pancakes.jpg}
        \captionsetup{width=0.9\linewidth}
        \captionof{figure}{A close photo of pancakes, small pieces of ice with raised edges from bumping one another. }
         \label{fig:pancake}
     \end{minipage}%
     \begin{minipage}{.5\textwidth}
         \centering
        \includegraphics[width=.9\linewidth]{cakes.jpg}
        \captionsetup{width=0.9\linewidth}
        \captionof{figure}{An example of a large ice cakes. For reference, the pictured ship, the R/V Sikuliaq, is 79.6 m in length.}
             \label{fig:cake}
      \end{minipage}
\end{figure}


\section{UAVs as a Platform for Remote Sensing}
\label{sec:remotesensing}
Remote sensing is the process of obtaining data without 
physical contact. The field of remote sensing, like most modern fields that 
are associated with data, has experienced an explosion in the amount, quality, and 
accessibility of data. 
Our images are from the visible spectrum and try to represent the world as they are seen 
with the human eye, however, often remote sensing is performed with other types of sensors. 
Passive sensors are also often used in other electromagnetic 
frequency bands like infrared or thermal bands for remote sensing. 
Passive sensors like cameras, sense energy that it has not itself emitted, while 
Active sensors emit energy and measure its reflectance. 
Popular active sensors like
RADAR (Radio Detection And Ranging) or LiDAR (LIght Detection and Ranging) are 
more energy intensive, but they can often show physical characteristics of the objects 
they are measuring. 
Historically  most remote sensing has been performed  from platforms 
such as satellites or manned aircraft.  
These systems 
observe the earth from a high altitude and thus have a lower
resolution when compared to UAVs flying close to the ground. 
The lower resolution significantly reduces the accuracy requirements for image
stitching accuracy. This means that there is little background from high altitude remote sensing 
that is relevant to low altitude mosaicing, but we will give a brief history 
of high altitude remote sensing for sea ice segmentation in Section \ref{sec:seg}.

Remote sensing has long been dominated by 
governments and commercial organizations, but recent 
advances in unmanned aerial vehicles have brought flight to 
the average consumer and researcher. 
Individual UAVs vary widely in terms of payload, endurance, weight, 
and speed and a comprehensive review is out of the scope of this thesis.

One important consideration for shipboard 
operations is the space needed for launch and recovery. There are two general 
types of architectures of UAVs: fixed wing and rotary winged. Fixed wing vehicles,
which are shaped like an airplane, are 
typically more stable and have longer range than rotary wing vehicles, which
propel themselves like a helicopter. However, fixed-wing vehicles have 
a more complicated deployment when compared to rotary-winged vehicles, 
requiring horizontal space for takeoff and landing. 
This can be difficult on a ship with limited deck space. Rotary-winged vehicles 
are generally more agile than fixed-wing aircraft and have less complicated 
landings and takeoffs. 

UAVs vary greatly 
in their computational capabilities and complexity. Many are capable of 
autonomous flight, while others rely solely on remote control by humans. For 
the purpose of this research, 
we disregard flight planning and control to focus only on the data products. 
See  \cite{Colomina14} by Colomina and Molina for an extensive review of UAV technology 
and sensors as we will only give a short introduction to sensors relevent to our 
work in the following paragraphs. 


\subsection{Sensors}
Nearly all systems will have some sensing capabilities necessary for navigation, 
though the accuracy and capabilities of these devices can vary widely. These systems  can 
have an enormous impact on the quality of flight and data products captured.

\subsubsection{Global Positioning System (GPS)}
The Global Positioning System, frequently shortened to GPS, is a system which 
provides the location and time when a receiver is within view 
of at least four system's satellites. A GPS receiver monitors  
the time of flight for messages passed between each of the visible satellites. 
The receiver uses this time of flight to perform a calculation of the its position, 
which is 
presented as latitude, longitude, 
and elevation from mean sea level. In our system, GPS points are the main method 
of UAV localization. Although the GPS provides frequent updates for the vehicle position, this calculation contains some amount of error that is almost always too large to perform image stitching at any reasonable resolution. GPS accuracy 
can vary depending on atmospheric effects and receiver quality, but generally 
location is reported correctly to within a few meters. 


\subsubsection{Inertial Measurement Unit (IMU)}
An inertial measurement unit (IMU) is a sensor that utilizes accelerometers, 
gyroscopes, and magnetometers situated orthogonally to each other to estimate the pitch, roll, and heading. 
This device provides important information  
 for both navigation (especially in the absence of GPS) and 
for determining how images captured from the UAV are oriented on the earth. 
Though IMUs are useful for rough estimates of pose, they tend to suffer from 
accumulated error which makes them undesirable for use for long term navigation.



\subsubsection{Digital Cameras}
Digital cameras have become ubiquitous for
capturing and recording the space around us. They are optical instruments 
which record light to memory using an image sensor.

There are several parameters that are important when considering cameras as a
sensor. The lens of the camera
serves the role of focusing light onto the light sensor and is quantified by 
the focal length, 
the distance between the lens and 
the image sensor. Image resolution is a measure of how much detail an image holds. 
A higher resolution means that an image 
sensor is able to better observe the smallest object for a given lens. 
A pixel is the smallest 
addressable element in the image and is used to quantify resolution. 
We typically refer to resolution by referring to the number of pixels in 
\emph{Rows x Columns} that are presented in am image. 
Although an image mosaic can be computed without any knowledge of 
the internal camera calibration parameters such as focal length, 
resolution, or relative camera motion between frames, these parameters 
are often necessary when relating imagery to physical features. This is necessary 
when relying on vision for navigation and obstacle avoidance. 


Moving platforms like UAVs which utilize cameras for mapping are typically configured 
as narrow-baseline or wide-baseline, where baseline refers to the distance between 
two cameras with overlapping scenes. 
Narrow-baseline, or simply 
stereo vision, involves mounting two cameras at a known distance from each other 
and capturing images simultaneously from both sensors.  This is advantageous in 
many situations because it allows us to calculate the offset between the same 
object captured in the two images
when compared to the known baseline between the cameras.  This setup can 
be used for 3D mapping and distance calculation of objects captured by both 
cameras \cite{Olson10}.
Wide-baseline, or monocular systems capture the scene from different positions 
using the same camera. Unlike stereo vision, monocular vision systems 
do not have an easy method of determining the distance to objects in photos 
from a single location and must rely on movement of the sensor to 
characterize objects from different viewpoints. Though narrow-baseline systems are attractive 
for  mapping nearby 3D scenes, they are fundamentally limited in their ability to measure 
depth by the distance of the baseline. 
This is because the accuracy of stereo degrades as the distance to the scene increases. 
In addition, a large baseline is not feasible on many
platforms, especially airborne systems. 

Rather than using two fixed cameras 
capturing the scene simultaneously as in narrow-baseline, we use a wide-baseline, monocular system.
Since the influential paper by Schmid and Mohr \cite{Schmid97} on wide-baseline stereo, 
many new algorithms have been introduced for making sense of this type of data. 
We will discuss this process more in Chapter \ref{chap:mosaic}. 


\section{Ice/Water Segmentation}
\label{sec:seg}
By far the most common technique for remotely monitoring sea ice is through data obtained 
by satellites. Most researchers work with data collected by Synthetic Aperture Radars (SARs), 
 a form of active microwave operated at around 5.3 GHz.
SAR is advantageous because it can operate without natural light and despite clouds to 
build models of the physical geometry of the ice that allows us to  measure age and thickness. 

\begin{figure}
\label{fig:radarsat}
\centering
\includegraphics[width=.9\linewidth, height=0.6\linewidth]{ice-glaces-canada.jpg}
  \captionsetup{width=.9\linewidth}
  \captionof{figure}{
 Example of ice classification and segmentation labels  over the Western Arctic captured with RADARSAT and produced by the Canadian Ice Service. RADARSAT-2 is the main SAR mission gathering data today \cite{CanadianIceService}
 }
\end{figure}

SAR datasets are typically captured  at a  relatively low resolution 
when compared to our data, with 1 pixel representing 
50 meters on the earth's surface with the data often plagued with speckle noise 
 \cite{CanadianIceService, Li15, Xu14, Clausi08}. Speckle noise is the 
 result of random reduction in the radar's return signal caused by interference 
 from objects on the scale of the wavelength. 
The Canadian Ice Service organizes and labels SAR data manually to provide low resolution 
ice maps. This labeling is an expensive and demanding task and  many attempts  have 
been made to automate the process, though none have been successfully implemented for public release \cite{Clausi03, Clausi08}. 
 In the following paragraph, we will summarize recent approaches to determine 
 the ice/water boundary in SAR imagery.


SAR sea ice segmentation approaches are typically pixel-based or 
texture-based.  A simple pixel-based grey threshold has been utilized quite 
successfully to separate ice and water, especially when combined with the watershed 
algorithm \cite{Haverkamp93, Clausi08}. 
The success of this thresholding is  because of the well-defined bimodal nature of sea ice imagery, 
however, this simple approach is also susceptible to speckle noise. 
Other pixel-based methods include mixture models \cite{Samadani95} and k-means
clustering \cite{Remund98, Clausi03}.
Markov Random Fields, described in \cite{Li95MRF} are used by Deng and Clausi
in \cite{Deng04, Deng05}, Yang and Clausi in \cite{Yang09} and by Clausi
in \cite{Clausi01, Clausi03}. MRFs base the pixel segmentation class of each pixel on 
its neighboring pixel values and on the conditional segment probabilities and
improve segmentation performance, especially in the presence of speckle noise. 

Texture-based algorithms, use features which are functions of groups of neighboring pixels, 
instead of individual pixels. This approach provides some robustness against speckle noise 
in SAR datasets. Many practitioners use 
the texture-based grey-level co-occurrence matrix (GLCM) which was introduced
by Haralick et al. \cite{Haralick73}. GLCMs are used extensively by others 
including Soh in \cite{Soh98} and Clausi \cite{Clausi01, Clausi03} on SAR data with demonstrated success.
Xu et al. \cite{Xu14} and Leigh et al. \cite{Leigh14} feeds texture 
features into kernel principle component analysis (KPCA)  and 
support vector machines respectively. 

%Convolutional neural networks, introduced in Section \ref{sec:relfeat} have  also been shown to perform well for remotely sensed images by. Basu et al. \cite{Basu15}. 
%In an innovative SAR paper, Li et al. in \cite{Li15} utilized the egg codes from 
%the Canadian Ice Service to label ice pixels using a \emph{Learning from 
%Label Proportions} (LFLP) approach to model patch-level ice vs water classification using the egg code percentages and overcoming a limitation of the labeling.  
% Wang et. al \cite{Wang16} demonstrated pixel-wise segmentation 
%to with absolute mean error of less than 10\%  when compared to human experts using deep CNNs in a small study of SAR images. 

%Manual of ICE
%https://www.ec.gc.ca/glaces-ice/default.asp?lang=En&n=2CE448E2-1

As discussed in Section \ref{sec:related}, Flaten and Zhang et al. introduced several papers  
(\cite{Flaten-thesis15, Zhang13, Zhang14, Zhang15floe}) for segmenting  high resolution 
images of sea ice that largely mirror the previous work in SAR imagery. 


\section{Vision Based Mapping}
\label{sec:map}
The problem of determining correspondence between two images is a well 
studied problem in computer vision.  
Most of the work and current research is found in the first step, as determining 
overlapping scenes from different viewpoints or in the face of drastic changes 
is still quite difficult.  Common changes in scene include 
differences in illumination, a rotated view, scaling, or moving objects. 
For a more thorough review of the  history and techniques for building image mosaics see work by Szeliski in \cite{Microsoft}, Prathap in \cite{Prathap16} and  Zitova in \cite{Zitova03}.

Constructing an image mosaic typically consists of several general steps:

\begin{enumerate}
    \item{\textbf{Determine images with overlapping views of a scene.} }
    \item{\textbf{Estimate the homography between images with overlapping scenes.}}
    \item{\textbf{Register images onto a common frame.}}
    \item{\textbf{Blend image seams to improve appearance.}}
\end{enumerate}
 
Many researchers and engineers have utilized UAVs to gather images to form large-scale 
photo mosaics \cite{Caballero07, Yahyanejad10, Cheng10, Pritt14, Kekec14, Prasad16, Vousdoukas11, Wang11, Wang14}.
Images collected with a UAV can be quickly, but roughly  aligned 
using data from sensors such as Global Positioning System (GPS) points and 
Inertial Measurement Unit (IMU) estimates.
However, these measurements are typically not 
accurate or reliable enough to generate a visually appealing image. 
Instead, most approaches for mosaicing UAV images use some form of a feature 
based approach to match images which we will discuss in the next few sections.

Problems tend to occur when there exists 
low-overlap between images, moving objects in the survey area, or featureless 
images. Our images contain all three of these problems and in the next few paragraphs, 
we give an overview of related work in these sub-fields of vision mapping. 



In addition to image mosaicing are two closely related objectives:
Structure from Motion (SFM) and Simultaneous Localization and Mapping (SLAM). Structure 
from motion \cite{Pollefeys98} is an extension of mosaicing that  produced  
3D geometry of the scene and the motion of the moving camera 
\cite{Morimoto97, Shum-3D98}.  
SFM have difficult scaling to long image sequences and fail without quality overlapping  points 
for triangulation. SFM is generally performed after all of the images have been collected, but SLAM performs a similar task but in real time. This allows a moving vehicle 
to both build a map of its environment and use this information for localization 
with an implementation of recursive Bayes estimation, usually an variant of a Kalman Filter or Particle Filter
\cite{MonoSLAM}. SLAM is currently irrelevant for our goal of
providing good estimates of sea ice in a setting which our platform necessarily 
has good localization through GPS. 




\subsection{Using Metadata to Refine Search Space}
Much of the current work in UAV image mosaicing relates to using metadata from 
the vehicle flight to improve mosaic output or SLAM and SFM efforts to determine 
vehicle location. For the purpose of this paper, we'll concentrate on advance made 
in improving the output map and not on localization. 

\begin{equation}
    \label{eq:ekf}
x_k = f(x_{k-1}, u_k) + w_k \\
z_k = h(x_k) + v_k
\end{equation}
The metadata provided by the vehicle's navigation sensors can provide low resolution clues to reduce the search space for matching correspondences. In \cite{Cheng10}, Xing et al.  divide image sequences from UAV flight into small groups of 
overlapping images to perform local optimization on. This optimization is based on the Extended Kalman 
Filter (EKF) and corrects homographies in a local area before performing a 
global optimization step on the entire image set. 
The EKF, developed for the Apollo Project \cite{NASA} and defined in Equation \ref{eq:ekf} extends the Kalman Filter to nonlinear systems by linearizing around  a working point. In the case of tracking features between frames, it allows us to estimate the location of these features given information about the state of the camera which is imperfectly measured by the UAV's sensors.  
Civera et al. \cite{Civera09} also 
use an EKF to improve the mosaic output from a rotating camera, but describes the process  from 
a SLAM perspective. This paper provided  real-time, drift free mosaicing, though it was only demonstrated in a small scene. 

Prasad, et al. investigated the problem of developing mosaics in quadcopter surveys 
with gaps in features using IMU information to improve convergence time.
The approach was demonstrated on building walls with only small regions (paintings) 
that provided keypoints\cite{Prasad16} 
Yahyanejad proposed a real-time approach of mosaicing images from UAVs for disaster 
response in \cite{Yahyanejad10}. First, the images were downsampled and roughly aligned using only 
GPS and IMU data the mosaic was transmitted to the operator from the UAV for real-time observation of the scene. Then as bandwidth and computational resources allowed, the alignment and images were refined. 
In \cite{Caballero07}, Caballero et al. a probabilistic framework which build uncertainty about feature matches into the mosaic building process. This approach  uses the match success between images to limit the complexity of the homography model selection thereby reducing a common failure in the mosaic. 

%Least median of squares (LMedS) is used for outlier rejection and a M-Estimator to compute the final result. This model is used if more than the 65\% of the matches are successfully tracked.

%Bay et al. \cite{Bay05} matched scenes using line segments extracted with a Canny 
%edge detector  in addition to descriptors to match homogeneous scenes like 
%architectural interiors. Line segments are able to convey geometrical and 
%topological information that made wide-baseline matching more reliable. 

%. Affine homography. If the percentage of success in thetrackingstepisbetween40\% and65\%,thenthe LMedS isnotused,giventhereductioninthenumber ofmatches.A relaxedM-Estimator(softpenalization) is carried out to compute the model.
%Euclidean homography. If the percentage is below 40\%, the set of data is too noisy and small to apply non- linear minimizations. The model is computed using least-squares.
%\cite{Caballero07}


\subsection{Dynamic Scenes}

There are many methods for mosaicing a scene with dynamics in the static mosaic
\cite{Fitzgibbon01, Rav-Acha07, Davis98, Uyttendaele01}. 
Some approaches eliminate all dynamic information in the scene, while others 
attempt to encapsulate the changes between images by overlaying the movement 
into the mosaic. 
However, our approach was to use a simpler, masking scheme to bar the feature 
extraction algorithm from extracting keypoints from the dynamic portions of our 
images. This also allows us to estimate sea ice coverage effectively solving 
two problems at once, so this portion of the project is further covered in \ref{chap:ice}. 
Our approach is similar to that of Vousdoukas in \cite{Vousdoukas11} who masked 
water from sea foam
for shorline images captured form an UAV. Vousdoukas
manually selected pixels associated with the 
sea foam and performed simple thresholding based on similar intensities before 
using SIFT descriptors to mosaic the coastline.



\chapter{Sea Ice Segmentation} 
\label{chap:ice}
To quantify ice floes it is advantageous to first segment the ice from the 
surrounding water and neighboring floes. This has been well studied for SAR 
imagery.
Much of the published work concerning the segmentation of sea 
ice from water is performed on data collected 
from synthetic aperture radar images collected from satellites. 
While this work is interesting 
and encouraging, the dataset from the low resolution SAR imagery and our high 
resolution images are different enough that there exists marginal overlap in 
performance for comparison. Nevertheless, the approaches used for SAR segmentation will be 
covered in the following section to provide background on approaches that have 
been proven to work on a similar dataset. 

%\begin{figure}
%     \centering
%\includegraphics[width=.9\linewidth, height=0.6\linewidth]{egg_code.jpg}
%  \captionsetup{width=.9\linewidth}
%  \captionof{figure}{
%  Ice estimates over the Hudson Bay region of Canada produced by the Canadian Ice Service \cite{CanadianIceService}. These reports are produced daily in season, subject to atmospheric and sea conditions. }
%    \label{fig:eggcode}
%\end{figure}
%
The watershed algorithm depends on the local minima  for the starting point. When
there are multiple local minima in the 
image, like is often the case in sea ice imagery, over-segmentation occurs. 

k-means clustering
\cite{MacQueen67} and demonstrated how k-means could be used to better segment


From this point on, we perform k-means clustering on the ice portion of the imagery. 


Our first aim in classification of sea ice is to isolate the water portions of each image from those portions with ice or snow. This can be though of as a segmentation problem. We also wish to further classify the ice portions of the image into classes of icebergs, slush, and cakes. 

Classification of an image into pixel classes can be thought of as either a 
supervised problem which requires training a set of classifiers from data  
which has been labelled at the pixel level or an unsupervised problem, which works without labels. For our problem of high resolution ice imagery, there does not exist (to our knowledge) a large database of labeled images from which to train in a supervised manner. Therefore, for this project, we developed a small  pixel-wise labelled dataset of  20 images $600x400$ pixels (TODO maybe add more) to serve as ground truth. 

We utilize the Jaccard index \cite{Jaccard1901} (also known as the Jaccard Similarity Coefficient) 
to qualify our results at the pixel level. The Jaccard index is a measure of the similarity between our calculated classification and the labelled pixel classes. This measure is defined by the 
size of the intersection divided by the size of the union of the dataset. 
The index is zero if there are no correctly classified pixels and 1 if all pixels are assigned the correct class. 

\begin{equation}
    \label{eq:jaccard}
    J(A,B) = \frac{|A \cap B|}{|A \cup B|}
\end{equation}
\section{Morphological Operation}
We developed a  baseline based on simple morphology operations. We first determine if the image contains water by counting the number of peaks in a histogram of the gray image over 255 bins. For all images observed in our dataset, images containing water has a clealy bimodal gray histogram. Therefore, for images with two distinct peaks, we select pixels contained within the first peak as water. We utilize a morphological open operation to remove small spaces that may
be influenced. 

\section{Texture Based Classification}


\chapter{Feature Descriptor Evaluation}
\label{chap:feat}
Like, most modern approaches, we register images by selecting key 
features from each image 
and measuring their similarity to key features in other images from the scene. 

There exist a wide variety of feature extraction and description
techniques that enable matching between visual correspondences despite 
rotation, scaling, or environmental changes such as lighting 
noise.  However, we found that for many of our images which contained only pictures of ice, very few keypoints 
could be located in the images (see Figures  \ref{fig:scale_bad} and \ref{fig:no_features}).  
This is because feature detectors rely on the existence of edges, corners, or  blobs to match images. If none 
occur, then no features can be detected. 
We wanted to develop an understanding of which approaches  work well with our type of data and under what 
circumstances algorithms failed and provide the results of several invariance 
tests in Section \ref{sec:featperformance} First, we give an describe the process and history of 
relating overlapping images in \ref{sec:relate} and introduce modern approaches in Section \ref{sec:whichfeat}. 


\section{Relating Overlapping Images}
\label{sec:relate}
Methods of determining image similarity generally falls into two camps:
direct and feature based approaches. Direct methods perform correlations or 
convolution operations over the prospective
images to measure their similarity. This involves 
a computation across all of the pixels in the image and tends to be
slow and memory intensive 
\cite{Lucas81, Barnea72, Szeliski95, Szeliski97, Shum-local98, Irani00}. 
However, direct methods are able to solve for motion of the 
camera and the correspondence of every pixel at the same time. Because this approach 
compares pixel values directly, it is prone to falter under illumination changes. 
The more modern approach is feature based..
Feature
based methods only measure the similarity of images based on 
interesting regions of the image, such as lines, blobs, or edges. 
The interesting regions are called \emph{features} 
or \emph{keypoints}. Keypoints within the image must be found algorithmically 
by a keypoint detection algorithm. Then the keypoint is described by a 
\emph{descriptor} using a descriptor algorithm.  Descriptors have been developed so 
that they are invariant to many of the transformations that plague images. 
By comparing images using only their interesting features, the motion of the 
camera can be solved for using only the
parts of the image in which correspondence is easy to detect. This reduces 
computational waste and improves accuracy.
The geometry calculated using the descriptors 
is then used to transform all of the pixels in 
the image. 

The feature based technique was first described by Schmid, who utilized Gaussian derivatives 
to describe a keypoint in a manner that was rotationally invariant \cite{Schmid97}.
Feature based methods are typically the modern approach because they are 
generally less computationally intensive than direct methods and because the 
feature descriptors can be developed to be invariant to image
transformations \cite{Schmid98, Lowe99, Brown03, Hu06, Elibol08, Rublee-ORB11, Alahi-FREAK12}. 
Like most modern methods, we utilize a feature based approach to 
determine image overlap. 
From this point onward in this paper, 
we will assume that overlap detection 
is performed using the feature based approach. 
Reliable  extraction of a manageable number of potentially 
corresponding features is a crucial prerequisite to successful mosaicing. The 
feature detector and descriptor choice is of significant importance. 
Relating the images 



\section{Overview of Descriptors}
\label{sec:whichfeat}
One of the most well known and often used keypoint descriptors is Scale
Invariant Feature Transforms (SIFT) \cite{Lowe04}, which was introduced by Lowe in 2004. SIFT descriptors detects keypoints 
based on Difference of Gaussians (DOG). 
This algorithm
has consistently been the top performer in terms of  
scale and rotation invariance, though it is has considerable overhead. 
Since the introduction of SIFT, many SIFT-like descriptors have been 
introduced to 
reduce computation time.  
Speeded-Up Robust Features (SURF) \cite{Bay-SURF08}, for instance,
yields similar feature performance to SURF, but with 
faster computation time. The performance improvement is 
accomplished by describing keypoints with the responses of Haar-like filters. 
Haar features are global-texture based. Both SIFT and SURF use an orientation 
operator.  Both SURF and SIFT use vector-based features, usually with a length of $128$ values that can be slow to compare. 
%Good foundation paper on multiscale approach \cite{Brown05} ??
%- this seemed to have a good descript of sift: http://www.cs.cmu.edu/~rahuls/pub/cvpr2004-keypoint-rahuls.pdf

%ORB
Although SURF achieves some speed gain over SIFT, it is still not realistic 
for real-time applications on platforms with limited capability \cite{Bekele13}. 
Binary keypoint descriptors aim to provide lightweight and fast computation with 
good performance by building the descriptor such that each bit is independent.
A binary descriptor uses significantly less memory (512 bits) for a given keypoint when compared with a gradient descriptor (usually 64 or 128 floating points) \cite{Prathap16}. 
This allows Hamming distance to be used rather than Euclidean distance to measure 
similarity and results in significant speed in matching with only marginal performance penalties \cite{Bekele13, Heinly12}. 
ORB \cite{Rublee-ORB11} or Oriented FAST and Rotated BRIEF descriptor is one of these computationally 
efficient binary descriptors that can be used in real-time for many applications. 
Like its name suggests, it works off of the very efficient FAST keypoint 
detector \cite{Rosten-FAST06} and BRIEF descriptor \cite{Calonder-BRIEF10}.
ORB overcomes the lack or rotational invariance in the BRIEF descriptor by 
de-correlating BRIEF features under rotational invariance, however, ORB is
only somewhat scale invariant.

The Fast REtinA keypoint, or FREAK \cite{Alahi-FREAK12}, descriptor 
introduced in 2012 by Alahi improves upon the fast Binary Robust Invariant Scalable Keypoints (BRISK) \cite{Leutenegger-BRISK11} descriptor and was 
inspired by the retinal pattern in the biological eye. 
FREAKs are in general faster to compute with lower memory load than 
 SIFT, SURF or BRISK. 


We also consider the Zernike descriptor which has been used successfully for 
matching images in low-contrast scenes \cite{Pizarro03}.  Zernike moment 
descriptors are a set of orthogonal polynomials that 
work well in the presence of 
 noise,  and large scale changes in a computationally efficient manner. 
 Only the magnitude of Zernike moments is rotationally invariant \cite{Pizarro04}.
We utilize the classic Harris keypoints \cite{Harris} for fast corner finding. 
Harris keypoints are detected at the local minima of an autocorrelation 
function that is invariant to geometric image transformation \cite{Schmid98}. 

Several new approaches utilize Convolutional Neural Networks (CNNs) to learn good features for matching images. CNNs 
have recently dominated many computer vision applications, outperforming methods 
which rely on hand-crafted features  \cite{Imagenet}. Simo-Serra et al. demonstrated a method for learning patch-level correspondence using a Siamese network of CNNs to serve as a drop-in replacement for traditional handcrafted 
feature descriptors like SIFT in \cite{deepdisc}. Their 128 length descriptor demonstrated consistent performance gains over state of the art. Though these features are slow to compute, significant speedup is seen on the GPU. They train the network matching $64x64$ gray-scale patches that were extracted from  stereo data. Although this approach seems to generalize well, it is also appealing in that one could potentially train customized feature extractors for a specific dataset.
In \cite{MountainView}, DeTone et al. also use a CNN, but instead of developing a descriptor for patches, 
they attempt to estimate an 8 degree of freedom homography between pairs of images directly using a 10 layer feedforward network. This approach is interesting in that they don't rely on traditional keypoint detectors or matching schemes. 



TODO: read and cite these
\cite{Chen10}
\cite{Hwang08}
\cite{Zhanlong13}
\cite{Badra98, Badra99}
\cite{Bin02}
Use this for equations
\cite{Ameyah07} 
\cite{Hwang08}
\cite{Chen10}

\section{Performance Evaluation}
\label{sec:featperformance}
Feature detecting algorithms tend to have different strengths and weaknesses as 
discussed in several studies \cite{Tuytelaars08, Mikolajczyk05, Heinly12, Bekele13}. 
Bekele in \cite{Bekele13} and Heinly in \cite{Heinly12}, in  particular, provided  good 
overviews of popular feature descriptors and their performance which guided our evaluation. 
As shown by Heinly in \cite{Heinly12}, the choice of keypoint detector and descriptor is dependent on the particular dataset. There is a tradeoff between 
computation time, memory load, invariance to rotation, illumination, or scale changes. To properly determine the best descriptor for a relatively unusual dataset, in  Chapter \ref{chap:mosaic} we evaluate different feature descriptors.
We also consider using multiple descriptors, as Alahi et al. showed in \cite{Alahi10} that using a grid of well performing descriptors achieve better matching than using a single one to match an image region. 


% Quantifying a match
We evaluate feature detectors and descriptors  
on a number of points: 

\begin{itemize}
    \item{Average number of keypoints}
    \item{Precision of matches}
    \item{Recall of matches} 
    \item{Average number of good matches}
    \item{Putative Match Ratio}
\end{itemize}

Precision is the number of correct matches 
out of the total matches found by a brute force matching algorithm for a particular image pair. The precision value greatly influences the performance of 
the homography estimation step of the mosaicing process because a large number of "bad" proposed matches will slow down the homography estimation or stall it altogether if there are too many bad "matches". 
A  perfect descriptor would give a recall equal to 1 for any precision


TODO plot precision-y vs recall curve 

\begin{equation}
    \textsf{matching score} = \frac{\textsf{number of correct matches} }{\textsf{number of features}}
\label{eq:matchingscore}
\end{equation}


\begin{equation}
    \textsf{recall} = \frac{\textsf{number of correct correspondences} }{\textsf{number of proposed correspondences}}
\label{eq:recall}
\end{equation}

\begin{equation}
    \textsf{1-precision} = \frac{\textsf{number of false correspondences} }{\textsf{number of proposed correspondences}}
\label{eq:precision}
\end{equation}

\begin{equation}
    \textsf{putative match ratio} = \frac{\textsf{number of correct correspondences }}{\textsf{number of features}}
\label{eq:putativematches}
\end{equation}



To measure the performance of each of the descriptors, we developed two subsets of our dataset: one for images with "good" features (ice/water boundaries) and 
one with few distinct features (slick, solid ice images) to compare performance across the algorithms. We then took each of the images and warped it in a prescribed way through rotation, scale change, or a variation in intensity. 
The results of this experiment are seen in Table \ref{tab:tableperf} and demonstrated in Figures \ref{fig:scale_bad} and \ref{fig:rot_good}. 

The putative match ratio, proposed by Heinly in \cite{Heinly12} and described by Equation \ref{eq:putativematches}, allows us to measure the \emph{selectivity} of the algorithm as it describes how many of the detected keypoints are eventually described as a match.  If many descriptors are highly similar, incorrect matches are likely to occur, decreasing the putative match ratio. 

Parameters in each of the detectors are 
tuned so that 4000 features are detected in the reference image.  
If more than 4000 features are detected in the image, only the first 4000 
are used in the evaluation.  


\begin{table}[h]
\caption{We show the median performance of various descriptors on the poor features dataset in the presence of scale, illumination, and rotations changes. }
\centering \noindent
            \begin{tabular}{| l | l || c | c | c | c | c |}
    \hline
    Dataset & Metric & SIFT & ORB & FREAK  & Zernike & Deepdesc \\
    \hline
    Poor Rotate & Num features & & & & & \\ 
    \hline
    Poor Rotate & Num Good Matches  & & & & & \\ 
    \hline
    Poor Rotate & Putative Match Ratio &  & & & & \\ 
    \hline
    Poor Rotate  & Precision & & & & & \\ 
    \hline
    Poor Rotate  & Recall &  & & & & \\ 
    \hline
    \hline
    Poor Scale & Num Features &  & & & & \\ 
    \hline
\end{tabular}
\label{tab:tableperf}
\end{table}



\begin{figure}
    \label{fig:test_demo}
    \centering
    \begin{minipage}{.5\textwidth}
        \centering
        \includegraphics[width=.9\linewidth,height=20cm]{orb_scale_bad.png}
        \captionsetup{width=0.9\linewidth}
        \captionof{figure}{An example of our evaluation of the ORB descriptor's response to scale change for an image with poor features. The algorithm finds few keypoints and is able to match few of them to the scaled image.}
         \label{fig:scale_bad}
     \end{minipage}%
     \begin{minipage}{.5\textwidth}
         \centering
        \includegraphics[width=.9\linewidth,height=20cm]{orb_rotate_good.png}
        \captionsetup{width=0.9\linewidth}
        \captionof{figure}{An example of our evaluation of the ORB descriptor's performance in when faced with rotation on an image with quality features. You can see that the approach is able to find many keypoints and match nearly all of them with the rotated image.  }
             \label{fig:rot_good}
      \end{minipage}
\end{figure}



%\begin{figure}
%\label{fig:rotate_results}
%\centering
%\includegraphics[width=\linewidth, height=.6\paperheight]{rotate.png}
%        \captionsetup{width=.9\linewidth}
%        \captionof{figure}{
%A comparison of different feature descriptors on two datasets of 60 images randomly selected over a variety of conditions. The dataset labeled "good features" contained images with strong corners and lines, while the dataset labeled "bad features" contained images with only snow or ice and not significant features. Each image was subjected to a change in rotation and scored on its ability to match to the original image using a brute-force matcher.  }
%\end{figure}
%
%\begin{figure}
%\label{fig:scale_results}
%\centering
%\includegraphics[width=\linewidth, height=.6\paperheight]{Scale.png}
%        \captionsetup{width=.9\linewidth}
%        \captionof{figure}{
%A comparison of different feature descriptors on two datasets of 60 images randomly selected over a variety of conditions. The dataset labeled "good features" contained images with strong corners and lines, while the dataset labeled "bad features" contained images with only snow or ice and not significant features. Each image was subjected to a change in scale and scored on its ability to match to the original image using a brute-force matcher.  }
%\end{figure}
%

%Percent of matches - quotient of dividing matches count on the minimum of keypoints count on two frames in percents.
%Percent of matches - quotient of dividing matches count on the minimum of keypoints count on two frames in percents.
%Percent of correct matches - quotient of dividing correct matches count on total matches count in percents.
%Matching ratio - percent of matches * percent of correct matches. In all charts i will use "Matching ratio" ( in percents) value for Y-axis.

%where high precision relates to a low false positive rate, and high recall relates to a low false negative rate. High scores for both show that the classifier is returning accurate results (high precision), as well as returning a majority of all positive results (high recall).
%

\chapter{Topology Estimation}
\label{chap:mosaic}

In the previous chapter, we discussed the process of selecting and describing an image 
by its dinstinctive features. In this chapter, we describe how to determine overlap based on these features. 
and disucuss the necessary requirements to register each image into a globally 
consistent and pleasing mosaic. 

\section{Matching Features}
After keypoints are extracted from two images, they must be compared for 
similarity to determine if the images contain regions of the same scene. 

Binary-valued feature descriptors (such as BRISK, ORB, or BRIEF) allow efficient 
matching when compared to 
vector-based features. Binary features can be quickly compared using the 
Hamming distance between descriptors. For Vector-based features such as SIFT 
and SURF, approximate nearest-neighbor search is used to match features. 
\begin{equation}
    \label{eq:euclidean}
    d(m, m') = \sqrt{(m_1-m'_1)^2 +  (m_2-m'_2)^2 + ... + (m_n-m'_n)^2}
\end{equation}


The Euclidean distance or $L^2$ Norm (Equation \ref{eq:euclidean} is used to calculate the distance between potential matches for vector-based descriptors like SURF and SIFT of length $n$. 
Hamming distance (Equation \ref{eq:hamming} is used for 
determining the distance between pairs of  binary descriptors like ORB and FREAK of length $n$. 
\begin{equation}
    \label{eq:hamming}
    d(m, m') = \sum_{k=0}^{k=n}{|m_k - m'_k|}
\end{equation}
Non brute force methods can also be used for approximate matching such as the 
Flann matcher and Approximate nearest neighbors. 
Maximum Likelihood
Least Median of Squares (LMS),
 TODO explain short methodsb
Incorrect matches, which are especially common among images with 
repetitive features, should be filtered out if possible. 
Lowe showed that if the two best matches were close in distance, then the 
probability of a false match was high \cite{Lowe04}. To combat this, he 
proposed a simple \emph{Ratio Test}
to measure the distance of the best match over the distance 
of the second best match and eliminated the match if it was greater than a threshold. 

\section{Defining the Homography}
After the corresponding pixels 
between overlapping images have been identified, the next step in the process 
of developing a mosaic is to estimate
the relationship between the images.
The correctness of the final mosaic is highly dependent on the 
accuracy of each of the calculated 
geometric relationship between overlapping images. 

The process of finding this relationship is called \emph{homography}. 
Homography is defined in Equation \ref{eq:homo}, where the  transformation matrix, $H$, maps points on one image, $i$, to corresponding points on the reference image or mosaic, $i'$. 
The transformations 
can be translation, similarity, affine, or projective which have 2, 4, 6, and 8 parameters respectively. 


\begin{equation}
x'=H*x
\label{eq:homo}
\end{equation}

\begin{equation}
x'=Ax+t
\label{eq:affine}
\end{equation}


\begin{equation}
    \label{eq:affineA}
       x' =
        \begin{bmatrix}
        a_{11} & a_{12} & a_{13} \\
        a_{21} & a_{22} & a_{23} \\
    \end{bmatrix}
        x 
\end{equation}

We will discuss each 
of these transformations in the following paragraphs as they are defined in 
\cite{SzeliskiBook}.  The most complex transformation,
a projective transform, only preserves straight lines and is the farthest an 
image can be transformed. This type of transformation is used in many mosaicing approaches, but can result in errors in instances with low overlap between 
images or when there are few distinctive features available \cite{Wang11}. We instead consider only affine tranformations which are more stable than projective transformations, 
though they are unable to 
explain 3D motions between image planes.
Parallel lines also remain parallel under the Affine transformation transformation. 
For our problem,  the 
errors of object height projection associated with the Affine transformation  are usually negligible because the distance from the camera to the scene is large and the scene itself if relatively flat. 
An affine transformation (defined in Equations \ref{eq:affine} and \ref{eq:affineA} describes scale, rotation, translation, and skew 
between two 2D planes in 2D space. In Equation \ref{eq:affineA}, A is written as an arbitrary $2x3$ matrix.  
If a UAV  were able to achieve a perfect downward-facing camera survey over flat terrain, a similarity transform 
which is defined by rotation, translation, and scale would be sufficient to describe the 
motion between images, though this is often not practical. 
Even simpler applications, such as a panorama building with a rotating camera, 
only need simple 2D translation and rotation to align images.  

Homography can be determined by various methods like RANSAC, Least Means Squares, or a Hough Transform. 
In this paper, we determine homography using  RANSAC or "RANdom SAmple 
Consensus. RANSAC \cite{RANSAC} is an iterative parameter estimation technique 
that is widely in image registration problems.  It works by repeatedly taking random 
pixel matches from a set of data, fitting them to a proposed model, and then measuring 
the error from that model against all matches. 
For each proposed model, all of the matches are classified as either inliers or outliers by calculating the residual 
error  with respect to the model.  
This process is performed until a maximum 
iteration is hit or a model is estimated which meets a stopping criteria. Then 
the best model (with the least residual) is fit for all of the data. 


The homography calculated from RANSAC 
is used to warp the image onto the reference plane which serves as the compositing surface. For our work, which involves a  largely translational camera,
a planar compositing 
surface is most appropriate. A planar surface preserves perspective so that straight lines 
in the scene are straight in the mosaic. Other applications, such as with a rotational camera as in a typical panorama is to use a cylindrical or spherical 
projection so that the surface is curved like a sphere. 


\section{Blending}

A pixel in the final stitched mosaic corresponds to a single point in the scene, 
but because that scene was captured at different times from a different
viewpoint, the pixels from the images being composited at that pixel may be 
different. 
This is especially true when neighboring images in a mosaic are captured at different points in time as illumination can vary significantly.  
To produce a seamless output mosaic, we must determine the best 
pixel value for each image in the overlapping region and for the pixels that border in these regions. There are many different approaches to selecting 
these pixel values and blending the competing images. 

One of the most popular methods for blending images together is through the 
use of multi-resolution splines \cite{Burt83}. This is called Pyramidal blending in which different bands of frequency in each of the images are mixed at different weights. The low-frequency color variations 
are blended smoothly, but the high frequency textures are blended quickly using 
each images Laplacian pyramid. In our case, homogeneous regions like ice sheets 
have low spatial frequency and are combined across a wide region.
Strong color changes, such as ice-water transitions show high spatial frequency 
and are fused over a small area. 


\chapter{Topology Estimation}

A globally consistent mosaic should use all overlap information to produce 
the best possible representation of the scene. 
Strictly local methods, which utilize no information about the camera position, but match neighboring images until there is 
only the final mosaic left, tend to accumulate errors 
resulting in distortion \cite{Lin07}. To reduce these accumulated errors, most approaches make an effort to reduce misregistration and distribute small errors 
across the entire mosaic to improve the global scene. 

We explore two methods for finding overlapping images in the dataset: the first which serves as our basis for comparison, the first simply performs a brute force 
search over all of the images in the dataset for overlap, the second exploits the UAV's metadata to pre-align images and reduce the feature search space. 
Historically, the most broadly used global refinement method solves for all
images transforms at once as a final computationally expensive step offline. 

The process of simultaneously 
aligning all of the images can be slow 
to converge, but generally results in a more correct mosaic than strictly local methods \cite{Pizarro-thesis03}. 
This method is neccessarily non-real time as all image are required for the computation. In this step, we require that each feature in a scene match to the same point 
in the final mosaic from all images for which the feature is included. This is often implemented through a process 
called \emph{bundle adjustment} \cite{bundleadjust}. Bundle adjustment  minimizes reprojection 
error  using a nonlinear 
least squares algorithm to solve for the transform parameters that minimize the 
distance between corresponding features in the mosaic. 

Alternative to the global approach, many modern image alignment problems prefer real-time or near real-time 
approaches that exploit metadata from the camera platform to ensure consistency.
This can take the form of simple pre-alignment to normalize rotation in images, 
or can be a more extensive filtering approach using all avaible pose information.
In \cite{Lin07}, Lin et al. proposed to register 
each frame in an image sequence to its previous image that had been 
previously referenced to a lower-scale map of the region. He was able to use
low resolution satellite imagery to improve global alignment high resolution 
UAV imagery in near real-time.  This resolves scale and rotation 
differences between the images before solving for pixel wise matching which 
prevents errors between frames from accumulating. 
Although an interesting approach, 
satellite imagery generally updates too slowly 
to be useful for the dynamic state of sea ice. 
Kekec at al uses Separating Axis Theorem (SAT), a technique borrowed from computer graphics in \cite{Kekec14} to match all features with an image added to a mosaic 



\section{Registration}
\subsection{Image Only Registration}
The non-metadata approach does exploit the fact that
 the UAV captured images in a spatial order as it was flying, 
so we attempt to match images with neighboring timestamps first.
This sequential approach is not necessary, but will decrease convergence time 
as it limits the number images that must be searched in a typical mission. 

\begin{itemize}
    \item{\emph{Incremental Links} initially solve for a global mosaic using 
        overlaps of the temporal sequence. The global mosaic is created and refined by adding 
    constraints as new overlaps become apparent \cite{Pizarro-thesis03}}
    \item{\emph{Incremental Mosaic} in which each new image in the sequence is 
        registered to the previous image and the matching points are stored. The 
        transforms of all new and previous images are calculated using all known overlaps. 
    }
\end{itemize}

\subsection{Pre-Align Registration}

Because we have a rough estimation of the pose of the camera when each image was captured, we have the ability to augment the data to pre-align or pre-scale images using vehicle heading and altitude to overcome a descriptor's poor performance. 

\subsection{Probabilistic Filtering Registration}

TODO describe  Filter and metadata search

As we show in Table \ref{tab:search}, exploiting the viewpoint from 
which the images were captured can greatly reduce the time required to find all overlaps.  

\begin{table}[h]
\caption{We show the performance of different search space }
\centering \noindent
    \begin{tabular}{| l | l || c | c | c | c | c |}
    \hline
    Flight 1 & Num Features &  & & & & \\ 
    \hline
\label{tab:search}
\end{tabular}
\end{table}


%Features with a low success ratio in matching (about 0.5) are deleted from the 
%map if at least 10 matches have been attempted. Mapmaintence allows deleting of non 
%trackable features - for instance those captured on moving objects such as 
%waves or people. 

%The percentage of successful matches obtained by the point tracker is used to have an estimation about the level ofthehierarchywherethehomographycomputationshould start.Thesepercentagethresholdswereobtainedempirically byprocessinghundredsofaerialimages.Eachlevelinvolves thefollowingdifferentsteps:
%


\chapter{Experiments and Results}
\label{chap:results}
\section{Experimental Setup}
Data used for this study was collected in the Beaufort Sea, approximately 
350 km north of Alaska's northern coastline (Figure \ref{fig:flightsites}) aboard the R/V Sikuliaq (Figure \ref{fig:shipscout}). The UAV performed 12 flights 
over 5 days from various locations at altitudes varying from TODO m to TODO m. 

TODO: total number of photos, total coverage area

\begin{figure}
     \centering
\includegraphics[width=.5\linewidth, height=0.5\linewidth]{flightsites.png}
  \captionsetup{width=.9\linewidth}
  \captionof{figure}{
  Data used for this study was collected in the Beaufort Sea, approximately 350 km north of Alaska's northern coastline.  }
    \label{fig:flightsites}
\end{figure}









\begin{figure}
    \centering
    \begin{minipage}{.5\textwidth}
        \centering
        \includegraphics[width=.9\linewidth,height=0.6\linewidth]{uav_snow_d.jpg}
        \captionsetup{width=0.9\linewidth}
        \captionof{figure}{The Phantom FPV Flying Wing EPO UAV used to gather data over sea ice. This vehicle was equipped with a TODO camera for capturing images. }
         \label{fig:ouruav}
     \end{minipage}%
     \begin{minipage}{.5\textwidth}
         \centering
        \includegraphics[width=.9\linewidth,height=0.6\linewidth]{launch_uav_d.jpg}
        \captionsetup{width=0.9\linewidth}
        \captionof{figure}{UAVs have small space requirements for both storage and flight, making them ideal scouts for working off of a ship.}
             \label{fig:shipscout}
      \end{minipage}
\end{figure}






\subsection{Flight Vehicle}
The vehicle used in our experiments is a fixed-wing UAV pictured in Figure \ref{fig:ouruav}. This vehicle has an endurance of 
approximately TODO. The vehicle has autonomous flight system which follows 
a specified trajectory over the earth's surface at a prescribed altitude. 

\subsection{Sensor Payload}
The vehicle caries a payload of low-cost IMU, GPS receiver and a 
downwards-mounted color camera. The GPS receiver computes earth-referenced position at TODO Hz with an accuracy of TODO. Photos are captured at a resolution 
of 6000x4000 pixels at a rate of TODO Hz. An on-board computer logs sensor data 
which is processed after the flight. 
Lens used is a Sony E  with a focal length of $20 mm$. 
Camera is a Sony ICLE-5100.
\section{Results}




\subsection{Validation - Synthetic Mosaic}
A synthetic survey is generated from a single image by dividing the image into
a grid overlapping sub-images.  This provides a planar scene with an ideal 
pinhole camera. Simple translations produce the resulting image. Another test 
introduced rotation and scaling into each sub-image. 
demo 
translation - minimal overlap, detected that there was no rotation or scaling.
rotation   -
-- show plot of actual angle of rotation vs calculated
scaling - images should be scaled to size of smallest before stitching
-- show plot of actual scaling vs calculated
demo 
\subsection{Flight Map}

\chapter{Conclusion and Future Work}
\label{chap:conclusion}
This paper has presented a framework for surveying sea ice with a UAV equipped with a color camera. This work opens the world of sea ice reconnaissance and 
research with its unique processing capability at a low cost. A classification algorithm is implemented to classify sea ice and a mosaicing routine is 
introduced to produce high-quality low-altitude maps. 

The final classification results indicate that the classifier performs well 
when distinguishing between sea and ice and between ice types.. 

Adaptive collection techniques to allow the UAV to search for areas that is has
not covered, or to prioritize coverage of interesting areas, such as regions with 
large icebergs. 
Iceberg elevation or size calculations

\pagebreak{}

\bibliography{biblio}
\begin{vita}
This should be a one-page short vita.

There can be more paragraphs.\end{vita}
\end{document}



%
%\begin{figure}[H]
%\noindent \begin{centering}
%\framebox{\begin{minipage}[t]{1\columnwidth}%
%\textbackslash{}documentclass{[}12pt,english{]}\{report\}
%
%\textbackslash{}usepackage\{UTSAthesis\}
%
%... use other packages ...
%
%\textbackslash{}begin\{document\}
%
%\textbackslash{}committee\{... \}
%
%\textbackslash{}informationitems\{... \}
%
%\textbackslash{}thesiscopyright\{...\}
%
%\textbackslash{}dedication\{\textbackslash{}emph\{I would like to
%dedicate this thesis/dissertation to ...\}\}
%
%\textbackslash{}title\{\textbackslash{}textbf\{First line\}\textbackslash{}\textbackslash{}
%\textbackslash{}textbf\{second line \}...\}
%
%\textbackslash{}author\{...\} 
%\textbackslash{}maketitle 
%\textbackslash{}begin\{acknowledgements\} ... \textbackslash{}end\{acknowledgements\}
%\textbackslash{}begin\{abstract\} ... \textbackslash{}end\{abstract\}
%\textbackslash{}newpage 
%\textbackslash{}pagenumbering \{arabic\} 
%\textbackslash{}setcounter \{page\}\{1\} 
%\textbackslash{}pagestyle\{plain\}
%\textbackslash{}chapter\{...\} \% or \textbackslash{}include\{chap3\}
%...
%\textbackslash{}singlespace
%\textbackslash{}bibliographystyle\{...\} 
%\textbackslash{}bibliography\{...\}
%\textbackslash{}begin\{vita\}...\textbackslash{}end\{vita\}%
%\end{minipage}}
%\par\end{centering}
%\caption{Structure of a thesis \protect\LaTeX{} file\label{fig:Structure-of-thesis}}
%\end{figure}
%
%
%The following commands are defined in UTSAthesis.sty and should be
%used in the order suggested in Fig. \ref{fig:Structure-of-thesis}
%to provide required format information.
%\begin{itemize}
%\item \textbackslash{}title\{Thesis Title\}. This can contain multiple lines.
%Use ``\textbackslash{}\textbackslash{}'' to go to the next line.
%\item \textbackslash{}author\{Name of Thesis Author\}
%\item \textbackslash{}thesiscopyright\{Optional Copyright Statement\} 
%\item \textbackslash{}dedication\{Optional Dedication\} 
%\item Either \textbackslash{}committee\{Supervisor Name, Degree\}\{Co-Supervisor
%or Committee B Name, Degree\}\{Committee C Name, Degree\}\{Committee
%D Name, Degree\}\{Committee E Name, Degree\} or the following commands
%separately.
%
%\begin{itemize}
%\item \textbackslash{}supervisor\{Supervisor Name, Degree\} 
%\item \textbackslash{}cosupervisor\{Co-Supervisor Name, Degree\} or \textbackslash{}committeeB\{Committe
%member B Name, Degree\} 
%\item \textbackslash{}committeeC\{Committe member C, Degree\} 
%\item \textbackslash{}committeeD\{Committe member D, Degree\} 
%\item \textbackslash{}committeeE\{Committe member E, Degree\}
%\end{itemize}
%\item Either \textbackslash{}informationitems\{Full Name of Degree\}\{Short
%Name of Degree\}\{Full Name of Department\}\{Full Name of College\}\{Month
%of Thesis\}\{Year of Thesis\} or use the following commands separately.
%
%\begin{itemize}
%\item \textbackslash{}degree\{Full Degree Name\} 
%\item \textbackslash{}degreeshort\{Short Degree Name\} 
%\item \textbackslash{}department\{Department Name\} 
%\item \textbackslash{}college\{College Name\} 
%\item \textbackslash{}thesismonth\{Month\} 
%\item \textbackslash{}thesisyear\{Year\} 
%\end{itemize}
%\item \textbackslash{}maketitle is the command to produce the signature
%page, copyright page, dedication page, and the title page. The position
%of this command is important. 
%\item \textbackslash{}begin\{acknowledgements\}
%
%
%People, organization, supports that you want to thank for 
%
%
%\textbackslash{}end\{acknowledgements\}
%
%\item \textbackslash{}begin\{abstract\}
%
%
%The abstract starts here. Should within one page.
%
%
%\textbackslash{}end\{abstract\} 
%
%\item The thesis/dissertation should then continue with chapters, appendixes,
%references. Before the first chapter, it is necessary to set Arabic
%page number. If the thesis/dissertation is long, it may be better
%to place chapters into separate \LaTeX{} files and include these sub-files
%using \textbackslash{}include\{\} command.
%\item \textbackslash{}begin\{vita\}
%
%
%The last item is a one-page curriculum vita
%
%
%\textbackslash{}end\{vita\}
%
%\end{itemize}
%
%\subsection{Produce the Outcome}
%
%To produce the pdf version of the thesis/dissertation, run pdflatex
%and bibtex.
%
%
%\section{The utsathesis.layout Package}
%
%The utsathesis.layout is an \LyX{} layout that provides a \LyX{} document
%layout for UTSA dissertation/thesis. This layout should be used together
%with the UTSAthesis.sty.
%
%
%\subsection{Installation}
%
%First, install UTSAthesis.sty as described in Section \ref{sec:UTSAthesis.sty}.
%Then, installed the \LyX{} on your system by following the instruction
%that comes with the \LyX{} package. Next, place the utsathesis.layout
%into your personal \LyX{} directory. On a Linux/Unix system, this
%directory is at \textasciitilde{}/.lyx/layouts. On Mac OS, it is at
%/User/<name>/Library/Application Support/\LyX{}-<version>/layouts.
%On Windows 7, it is at C:\textbackslash{}Users\textbackslash{}<name>\textbackslash{}AppData\textbackslash{}Roaming\textbackslash{}lyx<version>\textbackslash{}layouts.
%Remember to run Tools->Reconfigure inside \LyX{} to re-configure the
%system.
%
%
%\subsection{Use of utsathesis.layout Package}
%
%This document (sampleThesis.lyx) provides a template for using the
%utsathesis.layout to write a Ph.D. dissertation. For a Master's thesis,
%go to Document->Settings and set the class option to ms. Other important
%settings may include Document->Settings->\LaTeX{} Preamble, and the
%bibliography style.
%
%The document setting should be ``report (UTSAthesis 2012)''. The
%document should begin with committee info, thesis info, copyright,
%and dedication. These can be formatted using items in the FrontMatter
%in the pull-down menu. These should be followed by title, author,
%acknowledgments and the abstract. The placement and the order of these
%four items are important for generating the correctly formatted front
%pages of the thesis/dissertation. It is also important to add the
%``Start First Page'' item right before the first chapter. This item
%will set the correct page numbers for the main portion of the thesis/dissertation.
%
%At the end of the document, the ``Vita'' item in the BackMatter
%in the pull-down menu needs to be used to format a one-page vita.
%
%Regular chapters can be included in the main thesis document or more
%likely as sub-files, one per chapter. If sub-files are preferred,
%make sure the document settings of all sub-files are identical to
%the main document. 
%
%
%\chapter{Literature Review}
%
%We have some citations \cite{dabiri-optimization-isqed-2008,melhem-ieeetc-2003,pradhan-fault-tolerance-1986}.
%See the Bibliography for the format of references.
%
%\include{chapt3}
%
%
%\chapter{Solution and Evaluation}
%
%In this chapter, we show the structures of math formula, theorem commands,
%and floats (such as algorithm and table).
%
%
%\section{A Theory}
%\begin{defn}
%This is another definition.\end{defn}
%\begin{thm}
%This is a theorem.
%\begin{equation}
%X=\frac{AB}{Y}
%\end{equation}
%\end{thm}
%\begin{proof}
%The proof is done here.
%\end{proof}
%
%\section{An Algorithm}
%
%The following is the algorithm.
%
%\begin{algorithm}
%\begin{enumerate}
%\item Step One
%\item Step Two
%\end{enumerate}
%\caption{The Do-It-Yourself Method}
%
%
%\end{algorithm}
%
%
%
%\subsection{Evaluation}
%
%The evaluation results is shown in the following table. It is straightforward
%to place the caption of the table above or below the table.
%
%\begin{table}
%\caption{Evaluation Results}
%
%
%\noindent \centering{}%
%\begin{tabular}{|c|c|c|c|}
%\hline 
% & Method 1 & Method 2 & Method 3\tabularnewline
%\hline 
%\hline 
%Criterion 1 &  &  & \tabularnewline
%\hline 
%Criterion 2 &  &  & \tabularnewline
%\hline 
%Criterion 3 &  &  & \tabularnewline
%\hline 
%\end{tabular}
%\end{table}
%
%
%The following is a long table
%
%\noindent \begin{center}
%\begin{longtable}{|c|c|c|c|c|}
%\caption{A Long Table\label{tab:A-Long-Table}}
%\endfirsthead
%\multicolumn{5}{c}{\textbf{Table \ref{tab:A-Long-Table}}: Continued}\tabularnewline
%\endhead
%\hline 
%Column1 & Column 2 & Column 3 & Column 4 & Column 5\tabularnewline
%\hline 
%\hline 
%1 &  &  &  & \tabularnewline
%\hline 
%2 &  &  &  & \tabularnewline
%\hline 
%3 &  &  &  & \tabularnewline
%\hline 
%4 &  &  &  & \tabularnewline
%\hline 
%5 &  &  &  & \tabularnewline
%\hline 
%6 &  &  &  & \tabularnewline
%\hline 
%7 &  &  &  & \tabularnewline
%\hline 
%8 &  &  &  & \tabularnewline
%\hline 
%9 &  &  &  & \tabularnewline
%\hline 
%10 &  &  &  & \tabularnewline
%\hline 
%11 &  &  &  & \tabularnewline
%\hline 
%12 &  &  &  & \tabularnewline
%\hline 
%13 &  &  &  & \tabularnewline
%\hline 
%14 &  &  &  & \tabularnewline
%\hline 
%15 &  &  &  & \tabularnewline
%\hline 
%16 &  &  &  & \tabularnewline
%\hline 
%17 &  &  &  & \tabularnewline
%\hline 
%18 &  &  &  & \tabularnewline
%\hline 
%19 &  &  &  & \tabularnewline
%\hline 
%20 &  &  &  & \tabularnewline
%\hline 
%21 &  &  &  & \tabularnewline
%\hline 
%22 &  &  &  & \tabularnewline
%\hline 
%23 &  &  &  & \tabularnewline
%\hline 
%24 &  &  &  & \tabularnewline
%\hline 
%25 &  &  &  & \tabularnewline
%\hline 
%26 &  &  &  & \tabularnewline
%\hline 
%27 &  &  &  & \tabularnewline
%\hline 
%28 &  &  &  & \tabularnewline
%\hline 
%29 &  &  &  & \tabularnewline
%\hline 
%30 &  &  &  & \tabularnewline
%\hline 
%31 &  &  &  & \tabularnewline
%\hline 
%32 &  &  &  & \tabularnewline
%\hline 
%33 &  &  &  & \tabularnewline
%\hline 
%\end{longtable}
%\par\end{center}
%
%
%\chapter{Future Directions}
%
%There can be more chapters.
%
%\appendix
%
%\chapter{Notations }
%
%Here we show the use of multiple appendixes.
%
%
%\section{Math Notations}
%
%Each appendix can have sub-sections as a regular chapter.
%
%
%\section{Additional Notations}
%
%
%\chapter{Ontologies}
%
%These is another appendix.

%the individual components. 
