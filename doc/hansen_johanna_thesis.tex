%% LyX 2.0.5.1 created this file.  For more info, see http://www.lyx.org/.
%% Do not edit unless you really know what you are doing.
\documentclass[12pt,english]{report}
\usepackage[T1]{fontenc}
\usepackage[latin9]{inputenc}
\usepackage{babel}
\usepackage{longtable}
\usepackage{float}
\usepackage{calc}
\usepackage{amsthm}
\usepackage{amsmath}
\usepackage{setspace}
\usepackage[unicode=true,pdfusetitle,
 bookmarks=true,bookmarksnumbered=false,bookmarksopen=false,
 breaklinks=false,pdfborder={0 0 1},backref=false,colorlinks=false]
 {hyperref}

\makeatletter

%%%%%%%%%%%%%%%%%%%%%%%%%%%%%% LyX specific LaTeX commands.
\providecommand{\LyX}{\texorpdfstring%
  {L\kern-.1667em\lower.25em\hbox{Y}\kern-.125emX\@}
  {LyX}}
%% Because html converters don't know tabularnewline
\providecommand{\tabularnewline}{\\}
\floatstyle{ruled}
\newfloat{algorithm}{tbp}{loa}[chapter]
\providecommand{\algorithmname}{Algorithm}
\floatname{algorithm}{\protect\algorithmname}

%%%%%%%%%%%%%%%%%%%%%%%%%%%%%% Textclass specific LaTeX commands.
\usepackage{UTSAthesis}      
\usepackage{times}            
\usepackage{latexsym}

\newenvironment{ruledcenter}{%
  \begin{center}
  \rule{\textwidth}{1mm} } {%
  \rule{\textwidth}{1mm} 
  \end{center}}%


  \theoremstyle{definition}
  \newtheorem{defn}{\protect\definitionname}
\theoremstyle{plain}
\newtheorem{thm}{\protect\theoremname}

\@ifundefined{showcaptionsetup}{}{%
 \PassOptionsToPackage{caption=false}{subfig}}
\usepackage{subfig}
\makeatother

  \providecommand{\definitionname}{Definition}
\providecommand{\theoremname}{Theorem}

\begin{document}
\bibliographystyle{acm}
\committee{Sos Agaian, Ph.D., Chair}{Prof Hanumant Singh, Ph.D.}{Prof. C, Ph.D.}{Prof. D, Ph.D.}{Prof. E, Ph.D. }


\informationitems{Master of Science in Electrical Engineering}{M.Sc.}{Department of Electrical Engineering}{College of Engineering}{May}{ 2016 }


\thesiscopyright{Copyright 2016 Johanna Hansen \\
All rights reserved. }


\dedication{\emph{I would like to dedicate this thesis to the many mentors and 
teachers who have guided me throughout this process.}}


\title{\textbf{IceView: A System for Mosaiking Sea Ice using Zernike Shape Descriptors}}


\author{Johanna Hansen}
\maketitle
\begin{acknowledgements}

Thanks!

\end{acknowledgements}

\begin{abstract}
Unmanned Aerial Vehicles are providing revolutionary inexpensive, 
real-time access to the terrain around us. One application where this technology 
is especially valuable is for ships navigating in ice laden waters. For these ships, 
satellites are often too slow to provide up-to-date information and helicopters 
are cost prohibitive. 

Traditional image mosaiking algorithms tend to fail when faced with sea ice 
scenes for two reasons; ice itself typically has very few features from which 
to match to overlapping images and water pixels are dynamic in nature  
making it impossible to match with overlapping images that are taken at different 
points in time. The Iceview system attempts to solve these obstacles by 
utilizing multi-scale Harris point detector coupled with a Zernike feature 
descriptor.  This approach assumes that the extended scene 
is planar and determines homographies for each image by topology estimation 
through feature-based pairwise image registation across all images using a 
multi-scale Harris point detector with a feature descriptor. The approach is 
demonstrated using real data obtained on an Artic Expedition. 

\end{abstract}
\pageone{}

\chapter{Introduction}

\section{Motivation}

Expeditions into the worlds cold oceans requires expert navigation and 
real-time knowledge of the conditions. This region of the earth 
is both interesting from a scientific perspective because it has had little 
exploration and from a commercial aspect because of increasing 
accessibility to Artic and Antartic in recent years. 
However, sea ice presents a substantial challenge 
that requires constant monitoring. Historically, ships working in icey conditions 
have relied on human scouts, helicopters, and/or satellite imagery to monitor 
ice conditions, but these are suboptimal solutions. Human watches are obviously 
restricted by visibility in all but perfect conditions.
Manned aircraft such as helicopters are often prohibitively expensive and require 
expert pilots and precious deck space. Satellite imagery provides useful overviews, 
but is low-resolution, prone to occlusion by cloudcover or atmospheric conditions, 
is updated at a low frequency, and may not always be available. 

Unmanned Aerial Vehicles (UAVs) equipped with high
resolution digital cameras offer a low-cost, real-time alternitive to manned aircraft. 
They require little expertise to operate and little deck space. However, because 
there is no human observer of the terrain, the data collected from the must 
be processed and analyzed in a way that is useful for decision makers. 
Although UAVs can provide images with high resolution, each image only covers a 
small part of the interesting area.
 This means that a single flight can result in thousands 
of images that must be organized into a human-readable format. 
A process called image mosaiking must be performed to merge overlapping images 
of the terrain into a single high resolution image that is easy to comprehend and 
integrate into existing workflows. 
By registering the many images into a single mosaic, 
we can create large scenes of high resolution that are easy to integrate into 
existing Geographic Information Systems (GIS).
This map can be combined with other data sources such as weather, 
ship position, and current information to provide a useful overview of sea conditions.

Algorithms for aligning and stitching images into photomosaics have been 
widely studied and commercialized, however, traditional
methods tend to fail when tasked with stitching sea ice imagery.
Both the ice and sea that are present in this type of imagery present challenges 
for tradational mosaiking approaches. The ice itself is often nearly featureless, 
making it difficult for feature-based matching schemes to work robustly. 
Bodies of water are generally useless for feature matching because of its dynamic 
properties and meaningless features over time. 
This report approaches these problems by developing a pipeline for working 
with a seaice dataset. 


\section{Background}

Aerial photography has been used to gain a greater viewpoint of the world and 
has been used extensively reconassance, agriculture, and science. 
Historically, images were captured on air balloons and airplanes and then printed 
and aligned by hand. In modern times, digital cameras and smartphones are capable of 
automatically stitching panoramas on a small platform. 
This panorama is a rotational motion solved by a cylindrical or spherical surface model.
To produce a UAV image mosaic, however, the rotional motion about the camera axis 
is replaced by a translational motion model that changes point of view.
Mosaics with extended field of view benefit 
from the relatively large and equal distance to the scene in each image. 

Images can be aligned quickly with Global Positioning System (GPS) and 
Intertial Measurement Unit (IMU) data, however, these measurements are not 
accurate enough for a visually appealling image. In addition, accurate measurements 
would require more expensive sensors. 
Moreover, GPS is not always a available or reliable at extreme latitudes where 
ships are often working in sea ice.
Instead, most approaches for mosaiking UAV images is to extract common 
features between consecutive images to infer the motion between images.
This approach requires some degree of overlap between images and significantly 
higher computing requirments than a simple alignment based on sensors, but it 
produces much more accurate results. 

There exist many open-source and proprietary packages for developing mosaics 
from overlapping imagery such as Autopano or PhotoScan. However, experminents 
indicated that these packages were not reliable for developing mosaics with our 
data. Stitching and warping errors typically come from the accuracy of feature 
matching or the projection of object height. 

TODO: overview of air, underwater mosaic

\section{Thesis Overview}

The main problem is to determine overlapping regions between two 
neighboring images and determining the relative transformation
 between them in a process called image registration. 
 The two images are then aligned and merged or stitched together to form a larger image called a mosaic. 

The transformation between any two overlapping images can be described
linearly decomposed in to translation, rotation, and scaling components.

Translation, rotation, and scaling should be recovered at the same time.
Image stitching covers the method of combining many individual images into 
one mosaic. Determine how two images are aligned to each other. Blending t
Image registration strategies find correspondences between two images. 
Most methods utilize intersting points, such as regions of color, edges, 
contours, or distinctive points. These features are then searched for in 
nearby images. If distinctive features are detected, they are matched to 
determine correspondences. UAV images are taken at different points of view 
and at different times. 

Major errors come from the object projection onto the image. If the terrain to 
be surveyed is almost flat, the UAV images can be directly stitched. For these
near flat surfaces, the affine transformation can be directly used to translate 
image coordinates. \cite{Wang}


For ice imagery, we can take advantage of the 
large distance to the scene and the generally planar terrain surface 
to adopt a model which assumes equal distance to each object in the image.
This thesis demonstrates large-area 
mosaicing of sea-ice, addressing issues of image registration 
in a global framework. Relate images using a multi-scale feature detector with matching based on 
descriptors that provide invariance to rotation, scaling, and affine changes in 
intensity. This approach is purely image based and does not neccessitate the use 
of navigation data.

A set of features is computed for every image
feature tracking may fail if there is insufficient matches - low overlap or featureless.
Considerable effort has been invested in improving the robustness of 
image mosaic techniques to changes in sensors, illumination, and zoom.  

Image registration involves overlaying multiple images from the same scene 
that were captured from a subset of different viewpoints, times, and/or cameras.
This work attempts to build a standard system in which to mosaic images
with non-stationary objects and near featureless terrain, particularly sea ice.  

Because the distance from the UAV's camera is
much greater than the motion of the camera between views, a homographic model
can be use to describe the relationship between neighboring images 
\cite{Semple, Ma}. 

\begin{enumerate}
\item{Identify features or keypoints from ice regions}
\item{Determine matching features between images}
\item{Estimate homography between images that contain the same keypoints}
\item{Warp images according to estimated homographies so that their overlapping 
regions align}
\item{Paste warped images onto a common scene and blend neighboring pixels}
\end{enumerate}
 
For the iceview problem, we add an additional preliminary step to detect 
nonstationary features and remove them from the pixels that are considered for 
feature detection. 

TODO: talk about segmentation of water/ice

\chapter{Methods}
Unknown motion between images
Attitude and scale changes betweeen images with UAV


\section{Feature Extraction}

Image registration often use \emph{keypoints} or \emph{features} from an 
image and a \emph{descriptor} which contains a description of the neighborhood of 
pixels around a keypoint.  
Descriptors are used to pre-process images in order to make their 
features more invariant to scale, rotation, and translation transformations. 
Choosing an appropriate feature detection and description algorithm is a 
tradeoff between feature uniqueness, robustness, lighting, and 
computation time. 


There exists a wide variety of feature extraction algorithms.
One of the most well known and often used keypoint descriptors is Scale
 Invariant Feature Transform  or SIFT \cite{SIFT} which detects keypoints 
 based on Difference of Gaussians (DOG). 
 A successor to SIFT, Speeded-Up Robust Features (SURF) is a popular 
 algorithm that yields similar feature performance to SIFT, but with 
 faster computation time \cite{SURF}. 

Corner like features -Hessian and eigenvalue.
Harris corners are not invariant to scaling, and cross correlation
is not invariant to translation. 

Features can be extracted.... discrete features of 
interest as keypoints that are identified in different viewing conditions. 
The neighborhood around the keypoints is represented by a description vector. 
Theoretically, only four corresponding points are required to determine 
homography between two images. 

Features are used to determine which features come from corresponding locations 
in the images. Local motion areound each feature point is expected to be mostly 
translational. 
It is only necessary to match each image to a small number of neighboring 
images. 

Recall and precision can be used to compare performance of descriptors 
demonstrated.

Corner detectors are fast to find 
and describe, but more effort must be spent to match and reject incorrect 
matches. Sharp corners and edges make excellent features for detection 
and matching, this does not exist on images that are encompassed 
entirely be and iceberg. 
Evaluation of keypoint detectors. Bekele et al provides an evaluation 
of keypoint detectors in \cite{Bekele} in which escriptors are considered 
from BRIEF, ORB, BRISK, and FREAK. 

rotational invariant and can be made to be scale and translational invariant. 
Magnitude of zernike moments are rotationally invariant. Used by \cite{pizarro}
Zernike momemnt are a set of orthogonal polynomials that offer a complete 
solution to recover rotational and scaling parameters. 


Zernike Polynomials 
complex polynomials that are orthogonal over the unit circle $x^2 + y^2 = 1$. 
They can be expressed in radial and angular components. 

$
\label{eq:zernike polynomial}
V_{nm}(x,y) = V_{nm}(\rho, \theta) = R_{nm}(\rho)\emph{e}^{im\theta}
$
\emph{n} is a positive integer
\emph{m} is an integer such that $n-|m|$ is even and $|m|<=n$
$\rho$ is the magnitude of the vector from the origin to the point $(x,y)=x^2+y^2$
$\theta$ is the angle between the vector and the x-axis in a counter-clockwise direction



Two images contain brightness changes, viewpoint, rotation, scale.
To evaluate the descriptors on this dataset, four parameters were 
calculated for each of the proposed descriptors. Average number of keypoints, precision in percentage, recall in percentage, 
average number of best matches. Parameters in each of the detectors are 
tuned so that xxx features are detected in the reference image.  
-- What is the best number of features? 
If more than xxx features are detected, the first xxx are taken as 
in \cite{Bekele}.  

Calculation of the precision is performed by finding 
the ration of the number of matches that are output of RANSAC from each 
ratio test matches. 

Recall is calculated by taking the ratio of ransac 
matches to the number of keypoints. 

Accuracy is the ratio of the number of ransac matches to the number of 
keypoints. 
Average number of best matches - multiplying the average number of 
keypoints with the percentage of the accuracy. 

\section{Perspective Tranformation} 
The distance from the camera to the target object is much greater than the 
motion between the camera views, so a homographic model can be used to describe. 
 2-D projective transformation. Can be computed without any knowledge of 
the internal camera calibration parameters.  No need to know focal length, 
optical center, or relative camera motion between frames. 
Manually identify four or more corresponding points between two views, which
 is enough information to solve for eight unknowns in the 2-D projective
 transformation. 

An image 
transformation can be described by ... type of transformation matrix.
\ref{eq:General tranformation equation} where 
\emph{$s_i$} indicates scaling, 
\emph{$r_i$} rotation, and \emph{$t_i$} dictactes translation in the i-th direction. 
Thus, the problem can be reduced to recovering each component independently.


 $$
 \label{eq:General transformation equation}
 M =
 \begin{bmatrix}
 s_xr_{11}&s_xr_{12}&t_x \\ s_yr_{21}&s_yr_{22}&t_y \\ 0&0&1\\
 \end{bmatrix}
 $$

TODO: Discuss the history of each of the problems and show tests
Traditional matching algorithms recover motion models 

Once a match is determined between two images, each pixel from the source image 
must be mapped to a pixel coordinate in the destination image. 


 Ratio test? - ratio of the distance of the best match to the second best
 match - if this ratio is greater than 0.9 both of the matches are ignored. 
 If not, the best match is kept. This eliminates false matches. 
add new images to mosaic one at a time, aligning the most recent image with the 
previous ones already in the collection. A better alternative is to simultaneously 
align all the images together using a least squares framework to distribute 
mis-registration errors. The process of simultaneously adjusting pose parameters 
number of images in bundle adjustment. structure from motion problem
Bundle adjustment can solve for all camera parameters jointly. This minimizes
accumulated errors between pairwise homographies. 

Incremental links - solves for the global mosaic using the overlaps of the 
temporal sequence. 
 
 The disadvantage of bundle adjustment is that there are more variables to solve 
 for, so both each iteration and overal onvergence is slower. 

extract features and put into indexing structure
compare feature descriptors to find best match
ransac find set of inliers using pair of matches to hypothesize a similary

The covariance matrix of the homography is used to measure accuracy. 
RANdom Sample Consensus (RANSAC) was used to determine homography 
between two images.  Find the best projective relationship between the two 
image keypoints by computating a fundamental matrix from random sample 
points. Find the set of final best matches.

\section{Blending}
When merging two images into one image, blending of the pixel values should 
be applied to produce a more pleasing result. This approach utilizes the open 
source tool, enblend, that uses a multi-resolution spline to blend images 
together. Enblend combines the images across a transition zone that is proportional 
to the spatial frequency of the region. Homogenous regions like an ice sheet 
have low spatial and are combined across a wide region. Strong color changes, 
like a seal lying on an ice sheet, show high spatial frequency and are fused 
over a small area. 

\section{Assumptions and Approach}
Assume that images have been acquired in a temporal sequence. Although this is
not required, it will reduce convergence time. Navigation data
is not required. 

\section{Datasets}
\subsection{Aerial Surveys}

\subsection{Image Patches}

\chapter{Results}
\section{Error Metrics} 
Evaluation of quality - geospatial correctness of ground areas
standard measurments, cross correlation between images - 
speed ...
keypoint matching efficiency - 


\section{Validation - Synthetic Mosaic}
A synthetic survey is generated from a single image by dividing the image into
a grid overlapping sub-images.  This provides a planar scene with an ideal 
pinhole camera. Simple translations produce the resulting image. Another test 
should introduce rotation and scaling into each sub-image. 

\chapter{Conclusion and Future Work}
\subsection{UAV Sensor Integration}


\pagebreak{}
\section{References}
\bibliography{biblio}
\begin{vita}
This should be a one-page short vita.

There can be more paragraphs.\end{vita}
\end{document}



%
%\begin{figure}[H]
%\noindent \begin{centering}
%\framebox{\begin{minipage}[t]{1\columnwidth}%
%\textbackslash{}documentclass{[}12pt,english{]}\{report\}
%
%\textbackslash{}usepackage\{UTSAthesis\}
%
%... use other packages ...
%
%\textbackslash{}begin\{document\}
%
%\textbackslash{}committee\{... \}
%
%\textbackslash{}informationitems\{... \}
%
%\textbackslash{}thesiscopyright\{...\}
%
%\textbackslash{}dedication\{\textbackslash{}emph\{I would like to
%dedicate this thesis/dissertation to ...\}\}
%
%\textbackslash{}title\{\textbackslash{}textbf\{First line\}\textbackslash{}\textbackslash{}
%\textbackslash{}textbf\{second line \}...\}
%
%\textbackslash{}author\{...\} 
%
%\textbackslash{}maketitle 
%
%\textbackslash{}begin\{acknowledgements\} ... \textbackslash{}end\{acknowledgements\}
%
%\textbackslash{}begin\{abstract\} ... \textbackslash{}end\{abstract\}
%
%\textbackslash{}newpage 
%
%\textbackslash{}pagenumbering \{arabic\} 
%
%\textbackslash{}setcounter \{page\}\{1\} 
%
%\textbackslash{}pagestyle\{plain\}
%
%\textbackslash{}chapter\{...\} \% or \textbackslash{}include\{chap3\}
%
%...
%
%\textbackslash{}singlespace
%
%\textbackslash{}bibliographystyle\{...\} 
%
%\textbackslash{}bibliography\{...\}
%
%\textbackslash{}begin\{vita\}...\textbackslash{}end\{vita\}%
%\end{minipage}}
%\par\end{centering}
%
%\caption{Structure of a thesis \protect\LaTeX{} file\label{fig:Structure-of-thesis}}
%\end{figure}
%
%
%The following commands are defined in UTSAthesis.sty and should be
%used in the order suggested in Fig. \ref{fig:Structure-of-thesis}
%to provide required format information.
%\begin{itemize}
%\item \textbackslash{}title\{Thesis Title\}. This can contain multiple lines.
%Use ``\textbackslash{}\textbackslash{}'' to go to the next line.
%\item \textbackslash{}author\{Name of Thesis Author\}
%\item \textbackslash{}thesiscopyright\{Optional Copyright Statement\} 
%\item \textbackslash{}dedication\{Optional Dedication\} 
%\item Either \textbackslash{}committee\{Supervisor Name, Degree\}\{Co-Supervisor
%or Committee B Name, Degree\}\{Committee C Name, Degree\}\{Committee
%D Name, Degree\}\{Committee E Name, Degree\} or the following commands
%separately.
%
%\begin{itemize}
%\item \textbackslash{}supervisor\{Supervisor Name, Degree\} 
%\item \textbackslash{}cosupervisor\{Co-Supervisor Name, Degree\} or \textbackslash{}committeeB\{Committe
%member B Name, Degree\} 
%\item \textbackslash{}committeeC\{Committe member C, Degree\} 
%\item \textbackslash{}committeeD\{Committe member D, Degree\} 
%\item \textbackslash{}committeeE\{Committe member E, Degree\}
%\end{itemize}
%\item Either \textbackslash{}informationitems\{Full Name of Degree\}\{Short
%Name of Degree\}\{Full Name of Department\}\{Full Name of College\}\{Month
%of Thesis\}\{Year of Thesis\} or use the following commands separately.
%
%\begin{itemize}
%\item \textbackslash{}degree\{Full Degree Name\} 
%\item \textbackslash{}degreeshort\{Short Degree Name\} 
%\item \textbackslash{}department\{Department Name\} 
%\item \textbackslash{}college\{College Name\} 
%\item \textbackslash{}thesismonth\{Month\} 
%\item \textbackslash{}thesisyear\{Year\} 
%\end{itemize}
%\item \textbackslash{}maketitle is the command to produce the signature
%page, copyright page, dedication page, and the title page. The position
%of this command is important. 
%\item \textbackslash{}begin\{acknowledgements\}
%
%
%People, organization, supports that you want to thank for 
%
%
%\textbackslash{}end\{acknowledgements\}
%
%\item \textbackslash{}begin\{abstract\}
%
%
%The abstract starts here. Should within one page.
%
%
%\textbackslash{}end\{abstract\} 
%
%\item The thesis/dissertation should then continue with chapters, appendixes,
%references. Before the first chapter, it is necessary to set Arabic
%page number. If the thesis/dissertation is long, it may be better
%to place chapters into separate \LaTeX{} files and include these sub-files
%using \textbackslash{}include\{\} command.
%\item \textbackslash{}begin\{vita\}
%
%
%The last item is a one-page curriculum vita
%
%
%\textbackslash{}end\{vita\}
%
%\end{itemize}
%
%\subsection{Produce the Outcome}
%
%To produce the pdf version of the thesis/dissertation, run pdflatex
%and bibtex.
%
%
%\section{The utsathesis.layout Package}
%
%The utsathesis.layout is an \LyX{} layout that provides a \LyX{} document
%layout for UTSA dissertation/thesis. This layout should be used together
%with the UTSAthesis.sty.
%
%
%\subsection{Installation}
%
%First, install UTSAthesis.sty as described in Section \ref{sec:UTSAthesis.sty}.
%Then, installed the \LyX{} on your system by following the instruction
%that comes with the \LyX{} package. Next, place the utsathesis.layout
%into your personal \LyX{} directory. On a Linux/Unix system, this
%directory is at \textasciitilde{}/.lyx/layouts. On Mac OS, it is at
%/User/<name>/Library/Application Support/\LyX{}-<version>/layouts.
%On Windows 7, it is at C:\textbackslash{}Users\textbackslash{}<name>\textbackslash{}AppData\textbackslash{}Roaming\textbackslash{}lyx<version>\textbackslash{}layouts.
%Remember to run Tools->Reconfigure inside \LyX{} to re-configure the
%system.
%
%
%\subsection{Use of utsathesis.layout Package}
%
%This document (sampleThesis.lyx) provides a template for using the
%utsathesis.layout to write a Ph.D. dissertation. For a Master's thesis,
%go to Document->Settings and set the class option to ms. Other important
%settings may include Document->Settings->\LaTeX{} Preamble, and the
%bibliography style.
%
%The document setting should be ``report (UTSAthesis 2012)''. The
%document should begin with committee info, thesis info, copyright,
%and dedication. These can be formatted using items in the FrontMatter
%in the pull-down menu. These should be followed by title, author,
%acknowledgments and the abstract. The placement and the order of these
%four items are important for generating the correctly formatted front
%pages of the thesis/dissertation. It is also important to add the
%``Start First Page'' item right before the first chapter. This item
%will set the correct page numbers for the main portion of the thesis/dissertation.
%
%At the end of the document, the ``Vita'' item in the BackMatter
%in the pull-down menu needs to be used to format a one-page vita.
%
%Regular chapters can be included in the main thesis document or more
%likely as sub-files, one per chapter. If sub-files are preferred,
%make sure the document settings of all sub-files are identical to
%the main document. 
%
%
%\chapter{Literature Review}
%
%We have some citations \cite{dabiri-optimization-isqed-2008,melhem-ieeetc-2003,pradhan-fault-tolerance-1986}.
%See the Bibliography for the format of references.
%
%\include{chapt3}
%
%
%\chapter{Solution and Evaluation}
%
%In this chapter, we show the structures of math formula, theorem commands,
%and floats (such as algorithm and table).
%
%
%\section{A Theory}
%\begin{defn}
%This is another definition.\end{defn}
%\begin{thm}
%This is a theorem.
%\begin{equation}
%X=\frac{AB}{Y}
%\end{equation}
%\end{thm}
%\begin{proof}
%The proof is done here.
%\end{proof}
%
%\section{An Algorithm}
%
%The following is the algorithm.
%
%\begin{algorithm}
%\begin{enumerate}
%\item Step One
%\item Step Two
%\end{enumerate}
%\caption{The Do-It-Yourself Method}
%
%
%\end{algorithm}
%
%
%
%\subsection{Evaluation}
%
%The evaluation results is shown in the following table. It is straightforward
%to place the caption of the table above or below the table.
%
%\begin{table}
%\caption{Evaluation Results}
%
%
%\noindent \centering{}%
%\begin{tabular}{|c|c|c|c|}
%\hline 
% & Method 1 & Method 2 & Method 3\tabularnewline
%\hline 
%\hline 
%Criterion 1 &  &  & \tabularnewline
%\hline 
%Criterion 2 &  &  & \tabularnewline
%\hline 
%Criterion 3 &  &  & \tabularnewline
%\hline 
%\end{tabular}
%\end{table}
%
%
%The following is a long table
%
%\noindent \begin{center}
%\begin{longtable}{|c|c|c|c|c|}
%\caption{A Long Table\label{tab:A-Long-Table}}
%\endfirsthead
%\multicolumn{5}{c}{\textbf{Table \ref{tab:A-Long-Table}}: Continued}\tabularnewline
%\endhead
%\hline 
%Column1 & Column 2 & Column 3 & Column 4 & Column 5\tabularnewline
%\hline 
%\hline 
%1 &  &  &  & \tabularnewline
%\hline 
%2 &  &  &  & \tabularnewline
%\hline 
%3 &  &  &  & \tabularnewline
%\hline 
%4 &  &  &  & \tabularnewline
%\hline 
%5 &  &  &  & \tabularnewline
%\hline 
%6 &  &  &  & \tabularnewline
%\hline 
%7 &  &  &  & \tabularnewline
%\hline 
%8 &  &  &  & \tabularnewline
%\hline 
%9 &  &  &  & \tabularnewline
%\hline 
%10 &  &  &  & \tabularnewline
%\hline 
%11 &  &  &  & \tabularnewline
%\hline 
%12 &  &  &  & \tabularnewline
%\hline 
%13 &  &  &  & \tabularnewline
%\hline 
%14 &  &  &  & \tabularnewline
%\hline 
%15 &  &  &  & \tabularnewline
%\hline 
%16 &  &  &  & \tabularnewline
%\hline 
%17 &  &  &  & \tabularnewline
%\hline 
%18 &  &  &  & \tabularnewline
%\hline 
%19 &  &  &  & \tabularnewline
%\hline 
%20 &  &  &  & \tabularnewline
%\hline 
%21 &  &  &  & \tabularnewline
%\hline 
%22 &  &  &  & \tabularnewline
%\hline 
%23 &  &  &  & \tabularnewline
%\hline 
%24 &  &  &  & \tabularnewline
%\hline 
%25 &  &  &  & \tabularnewline
%\hline 
%26 &  &  &  & \tabularnewline
%\hline 
%27 &  &  &  & \tabularnewline
%\hline 
%28 &  &  &  & \tabularnewline
%\hline 
%29 &  &  &  & \tabularnewline
%\hline 
%30 &  &  &  & \tabularnewline
%\hline 
%31 &  &  &  & \tabularnewline
%\hline 
%32 &  &  &  & \tabularnewline
%\hline 
%33 &  &  &  & \tabularnewline
%\hline 
%\end{longtable}
%\par\end{center}
%
%
%\chapter{Future Directions}
%
%There can be more chapters.
%
%\appendix
%
%\chapter{Notations }
%
%Here we show the use of multiple appendixes.
%
%
%\section{Math Notations}
%
%Each appendix can have sub-sections as a regular chapter.
%
%
%\section{Additional Notations}
%
%
%\chapter{Ontologies}
%
%These is another appendix.

%\section{Image Segmentation}
%
%Image stitching techniques utilize features that are consistent amongst 
%overlapping images. Overlapping images captured from a UAV of the same scene are 
%captured at different points in time. If objects from the scene change between 
%the times in which the image is taken, the matching algorithm can fail. At best 
%this typically results in blurring of the scene which changes, at worst, this 
%can cause the overlapping images to not be matched at all. 
% field of view of the camera includes mostly dynamic regions,
%There are many methods for mosaicing a scene with dynamics in the static mosaic. 
%Some approaches eliminate all dynamic information in the scene, while others 
%attempt to encapsulate the changes between images by overlaying the movement 
%into the mosaic. For the purposes of the iceview system, the dynamic movement 
%of waves is not interesting information, so we blur the movement in this section. 
%We can exclude water from the feature extraction by first segmenting
%the image into ice or non-ice pixels. 
%
%Segmentation: 
%Thresholding gray level occurance
%gabor filter - texture
%glcp - statistical
%mrf
%
%The histograms for sea ice images are generally bimodal, so a mixture
%model can be used to approximate such a histogram and obtain
%the individual components. 
