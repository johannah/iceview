%% LyX 2.0.5.1 created this file.  For more info, see http://www.lyx.org/.
%% Do not edit unless you really know what you are doing.
\documentclass[12pt,english]{report}
\usepackage[T1]{fontenc}
\usepackage[latin9]{inputenc}
\usepackage{babel}
\usepackage{longtable}
\usepackage{float}
\usepackage{calc}
\usepackage{amsthm}
\usepackage{amsmath}
\usepackage{setspace}
\usepackage[unicode=true,pdfusetitle,
 bookmarks=true,bookmarksnumbered=false,bookmarksopen=false,
 breaklinks=false,pdfborder={0 0 1},backref=false,colorlinks=false]
 {hyperref}

\makeatletter

%%%%%%%%%%%%%%%%%%%%%%%%%%%%%% LyX specific LaTeX commands.
\providecommand{\LyX}{\texorpdfstring%
  {L\kern-.1667em\lower.25em\hbox{Y}\kern-.125emX\@}
  {LyX}}
%% Because html converters don't know tabularnewline
\providecommand{\tabularnewline}{\\}
\floatstyle{ruled}
\newfloat{algorithm}{tbp}{loa}[chapter]
\providecommand{\algorithmname}{Algorithm}
\floatname{algorithm}{\protect\algorithmname}

%%%%%%%%%%%%%%%%%%%%%%%%%%%%%% Textclass specific LaTeX commands.
\usepackage{UTSAthesis}      
\usepackage{times}            
\usepackage{latexsym}

\newenvironment{ruledcenter}{%
  \begin{center}
  \rule{\textwidth}{1mm} } {%
  \rule{\textwidth}{1mm} 
  \end{center}}%


  \theoremstyle{definition}
  \newtheorem{defn}{\protect\definitionname}
\theoremstyle{plain}
\newtheorem{thm}{\protect\theoremname}

\@ifundefined{showcaptionsetup}{}{%
 \PassOptionsToPackage{caption=false}{subfig}}
\usepackage{subfig}
\makeatother

  \providecommand{\definitionname}{Definition}
\providecommand{\theoremname}{Theorem}

\begin{document}



\committee{Sos Agaian, Ph.D., Chair}{Prof B, Ph.D.}{Prof. C, Ph.D.}{Prof. D, Ph.D.}{    Prof. E, Ph.D. }


\informationitems{Master of Science in Electrical Engineering}{M.Sc.}{Department of Electrical Engineering}{College of Engineering}{May}{ 2016 }


\thesiscopyright{Copyright 2016 Johanna Hansen \\
All rights reserved. }


\dedication{\emph{I would like to dedicate this thesis/dissertation template
to UTSA graduate students.}}


\title{\textbf{A Method for Mosaicing Sea Ice Imagery Collected by UAV }}


\author{Johanna Hansen}
\maketitle
\begin{acknowledgements}

I would also like to thank the UTSA Graduate School for reviewing
the outcome of this template document and correction of formatting
errors. 

\begin{singlespace}
\emph{This Masters Thesis/Recital Document or Doctoral Dissertation
was produced in accordance with guidelines which permit the inclusion
as part of the Masters Thesis/Recital Document or Doctoral Dissertation
the text of an original paper, or papers, submitted for publication.
The Masters Thesis/Recital Document or Doctoral Dissertation must
still conform to all other requirements explained in the Guide for
the Preparation of a Masters Thesis/Recital Document or Doctoral Dissertation
at The University of Texas at San Antonio. It must include a comprehensive
abstract, a full introduction and literature review, and a final overall
conclusion. Additional material (procedural and design data as well
as descriptions of equipment) must be provided in sufficient detail
to allow a clear and precise judgment to be made of the importance
and originality of the research reported. }

\emph{It is acceptable for this Masters Thesis/Recital Document or
Doctoral Dissertation to include as chapters authentic copies of papers
already published, provided these meet type size, margin, and legibility
requirements. In such cases, connecting texts, which provide logical
bridges between different manuscripts, are mandatory. Where the student
is not the sole author of a manuscript, the student is required to
make an explicit statement in the introductory material to that manuscript
describing the students contribution to the work and acknowledging
the contribution of the other author(s). The signatures of the Supervising
Committee which precede all other material in the Masters Thesis/Recital
Document or Doctoral Dissertation attest to the accuracy of this statement.}
\end{singlespace}
\end{acknowledgements}

\begin{abstract}
This is my abstract
\end{abstract}

\pageone{}


\chapter{Introduction}
Expeditions into the worlds cold oceans requires expert navigation and 
real-time knowledge of the conditions. Helicopters can be used to perform 
human led scouting expeditions from vessels locked in sea ice regions, however, these 
tools are costly to operate. Unmanned Aerial Vehicles (AUVs) equipped with high
resolution digital cameras offer a low-cost alternitive helicopters. The 
many small images that are collected must be geolocated so that the terrain
they cover can be interpreted by the ship's operators. Sea ice and large areas 
of water present a challenge because of their non-stationary nature.

\section{Motivation}

Although UAVs can provide images with high resolution, each images only cover
a small portion of the interesting area. To make the imagery human readable, 
we need to combine all of the sequential images together into a larger image 
that covers the surveyed area. Because the distance from the UAV's camera is
much greater than the motion of the camera between views, a homographic model
can be use to describe the relationship between neighboring images 
\cite{Semple, Ma}. 


\section{Overview of related work}
\section{Thesis Overview}

\chapter{Feature detection}
\section{Masking non-stationary segments}
\section{Scale-Space representation}
Invariant features - 
Harris corners are not invariant to scaling, and cross correlation
is not invariant to translation. 
\section{Interest points}
\section{Description of features}
\subsection{Zernike Moments}
\subsection{Learned Features}
\section{Image Matching}
Once features have been extracted from all images, they must be matches. 
It is only necessary to match each image to a small number of neighboring 
images. 
\section{Relating overlapping imagery}
Unknown motion between images

\subsection{Initial Match}
\subsection{Outlier Rejection}
RANSAC to select a set of inleiers that are compatible with a 
homography between the images. Next we apply a probablistic model to verify 
the match. 

Posterior probability that the match is correct using Baye's Rule

\chapter{Global Registration}
undle adjustment can solve for all camera parameters jointly. This minimizes
accumulated errors between pairwise homographies. 
\section{Assumptions and Approach}
\section{Reference Plane}
Attitude changes in the UAV may cause objects on neighboring images to not
be displayed with the same shape. Correction should be applied before calulating
the relationship between images. 

The first image of the flight is used as the refenece plane. Distortion is 
accumulated with the increasing number of sequence images. 
\section{Global Optimization}
After optimizing parameters within local areas, use bundle adjustment to 
improve all homographies locally. This step solves the problems of accumulated
errors. 

\section{Blending}

\chapter{Topology estimation}

\section{Assumptions and Approach}
Assume that images have been acquired in a temporal sequence. Although this is
not required, it will reduce convergence time. Navigation data
is not required. 

Incremental links - solves for the global mosaic using the overlaps of the 
temporal sequence. 


\section{Formulation}
Find the homographies that map every image onto the mosaic frame. 

\subsection{Pairwise registration}
\subsection{Global registration}
\section{Refinement}

\chapter{Results}
\section{Validation - Synthetic Mosaic}
A synthetic survey is generated from a single image by dividing the image into
a grid overlapping sub-images. 

\section{Aerial Surveys}

\section{References}


\pagebreak{}

\bibliographystyle{plain}
\bibitem{J. Semple, G. Kneebone. "Algebraic projective geometry," Oxford
University Press, 1952}

\bibitem{S. Ma, Z. Zhang. "Computer Vision," Science Press, Beijing, 2004.}
\bibliography{sampleThesis}

\begin{vita}
This should be a one-page short vita.

There can be more paragraphs.\end{vita}

\end{document}


%
%\begin{figure}[H]
%\noindent \begin{centering}
%\framebox{\begin{minipage}[t]{1\columnwidth}%
%\textbackslash{}documentclass{[}12pt,english{]}\{report\}
%
%\textbackslash{}usepackage\{UTSAthesis\}
%
%... use other packages ...
%
%\textbackslash{}begin\{document\}
%
%\textbackslash{}committee\{... \}
%
%\textbackslash{}informationitems\{... \}
%
%\textbackslash{}thesiscopyright\{...\}
%
%\textbackslash{}dedication\{\textbackslash{}emph\{I would like to
%dedicate this thesis/dissertation to ...\}\}
%
%\textbackslash{}title\{\textbackslash{}textbf\{First line\}\textbackslash{}\textbackslash{}
%\textbackslash{}textbf\{second line \}...\}
%
%\textbackslash{}author\{...\} 
%
%\textbackslash{}maketitle 
%
%\textbackslash{}begin\{acknowledgements\} ... \textbackslash{}end\{acknowledgements\}
%
%\textbackslash{}begin\{abstract\} ... \textbackslash{}end\{abstract\}
%
%\textbackslash{}newpage 
%
%\textbackslash{}pagenumbering \{arabic\} 
%
%\textbackslash{}setcounter \{page\}\{1\} 
%
%\textbackslash{}pagestyle\{plain\}
%
%\textbackslash{}chapter\{...\} \% or \textbackslash{}include\{chap3\}
%
%...
%
%\textbackslash{}singlespace
%
%\textbackslash{}bibliographystyle\{...\} 
%
%\textbackslash{}bibliography\{...\}
%
%\textbackslash{}begin\{vita\}...\textbackslash{}end\{vita\}%
%\end{minipage}}
%\par\end{centering}
%
%\caption{Structure of a thesis \protect\LaTeX{} file\label{fig:Structure-of-thesis}}
%\end{figure}
%
%
%The following commands are defined in UTSAthesis.sty and should be
%used in the order suggested in Fig. \ref{fig:Structure-of-thesis}
%to provide required format information.
%\begin{itemize}
%\item \textbackslash{}title\{Thesis Title\}. This can contain multiple lines.
%Use ``\textbackslash{}\textbackslash{}'' to go to the next line.
%\item \textbackslash{}author\{Name of Thesis Author\}
%\item \textbackslash{}thesiscopyright\{Optional Copyright Statement\} 
%\item \textbackslash{}dedication\{Optional Dedication\} 
%\item Either \textbackslash{}committee\{Supervisor Name, Degree\}\{Co-Supervisor
%or Committee B Name, Degree\}\{Committee C Name, Degree\}\{Committee
%D Name, Degree\}\{Committee E Name, Degree\} or the following commands
%separately.
%
%\begin{itemize}
%\item \textbackslash{}supervisor\{Supervisor Name, Degree\} 
%\item \textbackslash{}cosupervisor\{Co-Supervisor Name, Degree\} or \textbackslash{}committeeB\{Committe
%member B Name, Degree\} 
%\item \textbackslash{}committeeC\{Committe member C, Degree\} 
%\item \textbackslash{}committeeD\{Committe member D, Degree\} 
%\item \textbackslash{}committeeE\{Committe member E, Degree\}
%\end{itemize}
%\item Either \textbackslash{}informationitems\{Full Name of Degree\}\{Short
%Name of Degree\}\{Full Name of Department\}\{Full Name of College\}\{Month
%of Thesis\}\{Year of Thesis\} or use the following commands separately.
%
%\begin{itemize}
%\item \textbackslash{}degree\{Full Degree Name\} 
%\item \textbackslash{}degreeshort\{Short Degree Name\} 
%\item \textbackslash{}department\{Department Name\} 
%\item \textbackslash{}college\{College Name\} 
%\item \textbackslash{}thesismonth\{Month\} 
%\item \textbackslash{}thesisyear\{Year\} 
%\end{itemize}
%\item \textbackslash{}maketitle is the command to produce the signature
%page, copyright page, dedication page, and the title page. The position
%of this command is important. 
%\item \textbackslash{}begin\{acknowledgements\}
%
%
%People, organization, supports that you want to thank for 
%
%
%\textbackslash{}end\{acknowledgements\}
%
%\item \textbackslash{}begin\{abstract\}
%
%
%The abstract starts here. Should within one page.
%
%
%\textbackslash{}end\{abstract\} 
%
%\item The thesis/dissertation should then continue with chapters, appendixes,
%references. Before the first chapter, it is necessary to set Arabic
%page number. If the thesis/dissertation is long, it may be better
%to place chapters into separate \LaTeX{} files and include these sub-files
%using \textbackslash{}include\{\} command.
%\item \textbackslash{}begin\{vita\}
%
%
%The last item is a one-page curriculum vita
%
%
%\textbackslash{}end\{vita\}
%
%\end{itemize}
%
%\subsection{Produce the Outcome}
%
%To produce the pdf version of the thesis/dissertation, run pdflatex
%and bibtex.
%
%
%\section{The utsathesis.layout Package}
%
%The utsathesis.layout is an \LyX{} layout that provides a \LyX{} document
%layout for UTSA dissertation/thesis. This layout should be used together
%with the UTSAthesis.sty.
%
%
%\subsection{Installation}
%
%First, install UTSAthesis.sty as described in Section \ref{sec:UTSAthesis.sty}.
%Then, installed the \LyX{} on your system by following the instruction
%that comes with the \LyX{} package. Next, place the utsathesis.layout
%into your personal \LyX{} directory. On a Linux/Unix system, this
%directory is at \textasciitilde{}/.lyx/layouts. On Mac OS, it is at
%/User/<name>/Library/Application Support/\LyX{}-<version>/layouts.
%On Windows 7, it is at C:\textbackslash{}Users\textbackslash{}<name>\textbackslash{}AppData\textbackslash{}Roaming\textbackslash{}lyx<version>\textbackslash{}layouts.
%Remember to run Tools->Reconfigure inside \LyX{} to re-configure the
%system.
%
%
%\subsection{Use of utsathesis.layout Package}
%
%This document (sampleThesis.lyx) provides a template for using the
%utsathesis.layout to write a Ph.D. dissertation. For a Master's thesis,
%go to Document->Settings and set the class option to ms. Other important
%settings may include Document->Settings->\LaTeX{} Preamble, and the
%bibliography style.
%
%The document setting should be ``report (UTSAthesis 2012)''. The
%document should begin with committee info, thesis info, copyright,
%and dedication. These can be formatted using items in the FrontMatter
%in the pull-down menu. These should be followed by title, author,
%acknowledgments and the abstract. The placement and the order of these
%four items are important for generating the correctly formatted front
%pages of the thesis/dissertation. It is also important to add the
%``Start First Page'' item right before the first chapter. This item
%will set the correct page numbers for the main portion of the thesis/dissertation.
%
%At the end of the document, the ``Vita'' item in the BackMatter
%in the pull-down menu needs to be used to format a one-page vita.
%
%Regular chapters can be included in the main thesis document or more
%likely as sub-files, one per chapter. If sub-files are preferred,
%make sure the document settings of all sub-files are identical to
%the main document. 
%
%
%\chapter{Literature Review}
%
%We have some citations \cite{dabiri-optimization-isqed-2008,melhem-ieeetc-2003,pradhan-fault-tolerance-1986}.
%See the Bibliography for the format of references.
%
%\include{chapt3}
%
%
%\chapter{Solution and Evaluation}
%
%In this chapter, we show the structures of math formula, theorem commands,
%and floats (such as algorithm and table).
%
%
%\section{A Theory}
%\begin{defn}
%This is another definition.\end{defn}
%\begin{thm}
%This is a theorem.
%\begin{equation}
%X=\frac{AB}{Y}
%\end{equation}
%\end{thm}
%\begin{proof}
%The proof is done here.
%\end{proof}
%
%\section{An Algorithm}
%
%The following is the algorithm.
%
%\begin{algorithm}
%\begin{enumerate}
%\item Step One
%\item Step Two
%\end{enumerate}
%\caption{The Do-It-Yourself Method}
%
%
%\end{algorithm}
%
%
%
%\subsection{Evaluation}
%
%The evaluation results is shown in the following table. It is straightforward
%to place the caption of the table above or below the table.
%
%\begin{table}
%\caption{Evaluation Results}
%
%
%\noindent \centering{}%
%\begin{tabular}{|c|c|c|c|}
%\hline 
% & Method 1 & Method 2 & Method 3\tabularnewline
%\hline 
%\hline 
%Criterion 1 &  &  & \tabularnewline
%\hline 
%Criterion 2 &  &  & \tabularnewline
%\hline 
%Criterion 3 &  &  & \tabularnewline
%\hline 
%\end{tabular}
%\end{table}
%
%
%The following is a long table
%
%\noindent \begin{center}
%\begin{longtable}{|c|c|c|c|c|}
%\caption{A Long Table\label{tab:A-Long-Table}}
%\endfirsthead
%\multicolumn{5}{c}{\textbf{Table \ref{tab:A-Long-Table}}: Continued}\tabularnewline
%\endhead
%\hline 
%Column1 & Column 2 & Column 3 & Column 4 & Column 5\tabularnewline
%\hline 
%\hline 
%1 &  &  &  & \tabularnewline
%\hline 
%2 &  &  &  & \tabularnewline
%\hline 
%3 &  &  &  & \tabularnewline
%\hline 
%4 &  &  &  & \tabularnewline
%\hline 
%5 &  &  &  & \tabularnewline
%\hline 
%6 &  &  &  & \tabularnewline
%\hline 
%7 &  &  &  & \tabularnewline
%\hline 
%8 &  &  &  & \tabularnewline
%\hline 
%9 &  &  &  & \tabularnewline
%\hline 
%10 &  &  &  & \tabularnewline
%\hline 
%11 &  &  &  & \tabularnewline
%\hline 
%12 &  &  &  & \tabularnewline
%\hline 
%13 &  &  &  & \tabularnewline
%\hline 
%14 &  &  &  & \tabularnewline
%\hline 
%15 &  &  &  & \tabularnewline
%\hline 
%16 &  &  &  & \tabularnewline
%\hline 
%17 &  &  &  & \tabularnewline
%\hline 
%18 &  &  &  & \tabularnewline
%\hline 
%19 &  &  &  & \tabularnewline
%\hline 
%20 &  &  &  & \tabularnewline
%\hline 
%21 &  &  &  & \tabularnewline
%\hline 
%22 &  &  &  & \tabularnewline
%\hline 
%23 &  &  &  & \tabularnewline
%\hline 
%24 &  &  &  & \tabularnewline
%\hline 
%25 &  &  &  & \tabularnewline
%\hline 
%26 &  &  &  & \tabularnewline
%\hline 
%27 &  &  &  & \tabularnewline
%\hline 
%28 &  &  &  & \tabularnewline
%\hline 
%29 &  &  &  & \tabularnewline
%\hline 
%30 &  &  &  & \tabularnewline
%\hline 
%31 &  &  &  & \tabularnewline
%\hline 
%32 &  &  &  & \tabularnewline
%\hline 
%33 &  &  &  & \tabularnewline
%\hline 
%\end{longtable}
%\par\end{center}
%
%
%\chapter{Future Directions}
%
%There can be more chapters.
%
%\appendix
%
%\chapter{Notations }
%
%Here we show the use of multiple appendixes.
%
%
%\section{Math Notations}
%
%Each appendix can have sub-sections as a regular chapter.
%
%
%\section{Additional Notations}
%
%
%\chapter{Ontologies}
%
%These is another appendix.

